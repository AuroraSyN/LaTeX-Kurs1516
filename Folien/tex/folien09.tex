% !TEX program = lualatex
% !TEX spellcheck = de_DE

\documentclass{beamer}

\newif\ifweb
\webfalse
\webtrue

\usepackage{
	dtklogos,
	fontspec,
	graphicx,
	hologo,
	mathtools,
	microtype,
	pdfmarginpar,
	polyglossia,
	qrcode,
	shortvrb,
	showexpl,
	siunitx,
	tikz,
	pgfplots,
	xspace,
}

\usepackage{
	booktabs,
	blindtext,
	subfloat,
	wrapfig,
	floatflt,
	picinpar,
}


\setmainlanguage{german}

\setsansfont{Linux Biolinum O}
\setromanfont{Linux Libertine O}
\setmonofont[Scale=.95,AutoFakeSlant]{Inconsolata}
%\setmonofont[Scale=.99]{Anonymous Pro}

%\renewcommand{\sfdefault}{\rmdefault}


%% fancy-quotes %%%%%%%%%%%%%%
	\usepackage{tikz}
	\newcommand*\openquote{\makebox(25,-22){\scalebox{7}{\fontspec{Linux Biolinum O}“}}}
	\newcommand*\closequote{\makebox(25,-22){\scalebox{7}{\fontspec{Linux Biolinum O}”}}}
	\newcommand*{\OpenQuote}{\tikz[remember picture,overlay,xshift=-15pt,yshift=-10pt]
	     \node (OQ) {\openquote};}
	\newcommand*{\CloseQuote}{\tikz[remember picture,overlay,xshift=15pt,yshift=10pt]
	     \node (CQ) {\closequote};}
	\newenvironment{fancyquote}%
	{\hspace{-1em}\begin{quote}\OpenQuote\vspace*{1ex}\\\hspace*{1em}\begin{minipage}{.835\textwidth}}
	{\end{minipage}\vspace*{3.8ex}\\\hfill\CloseQuote\end{quote}}
	\newcommand\quoted[1]{\null\hfill{\scriptsize\sf #1}}
%%%%%%%%%%%%%%%%%%%%%%%%%%%%%%

\ifweb\else\renewcommand{\pdfmarginpar}[2][]{\null}\fi

%% overleaf %%%%%%%%%%%%%%%%%%
\usetikzlibrary{calc}
\tikzset{ href node/.style={alias=sourcenode,append after command={let \p1 = (sourcenode.north west),  \p2=(sourcenode.south east),\n1={\x2-\x1},\n2={\y1-\y2} in node [inner sep=0pt, outer sep=0pt,anchor=north west,at=(\p1)] {\href{#1}{\phantom{\rule{\n1}{\n2}}}}}}} % http://tex.stackexchange.com/a/36111
\newcommand{\overleaf}[1]{
	\begin{tikzpicture}[remember picture,overlay]
		\node [xshift=-1.2cm,yshift=1.5cm, href node={http://polr.me/#1}] at (current page.south east)
		{
			\scalebox{.55}{\parbox{4.1cm}{
				In Overleaf ausprobieren:\\[.8ex]
				\qrcode[height=4cm]{http://polr.me/#1}\\[1ex]
				\url{http://polr.me/#1}
			}}
		};
	\end{tikzpicture}
}
%%%%%%%%%%%%%%%%%%%%%%%%%%%%%%
\usetikzlibrary{shapes, arrows, trees}

\pgfplotsset{
	compat=1.12,
	width=7cm,
	lua backend=true,
}

\tikzset{onslide/.code args={<#1>#2}{%
  \only<#1>{\pgfkeysalso{#2}} % \pgfkeysalso doesn't change the path
}}
	
\newcommand{\meta}[1]{\textcolor{gray}{$\langle$\texttt{\textsl{#1}}$\rangle$}}
\newcommand{\pkg}[1]{\href{http://ctan.org/pkg/#1}{\alert{\texttt{#1}}}}
\newcommand{\TikZ}{Ti\textit{k}Z}
\newcommand{\BibTeX}{Bib\TeX}


\newenvironment{olcol}{
	\begin{columns}\begin{column}{.8\textwidth}
}{
	\end{column}\begin{column}{.108\textwidth}\relax\end{column}\end{columns}
}

\hypersetup{%
  unicode=true,
  pdfborder={000},
  pdftitle={Einführung in das Textsatzsystem LaTeX},
  pdfauthor={Moritz Brinkmann with credit to Arno Trautmann},
}

\lstloadlanguages{TeX}
\lstset{%
	language=TeX,
	backgroundcolor=\color[RGB]{229, 229, 239},
	basicstyle=\ttfamily\small,
	breakindent=0em,
	breaklines=true,
	commentstyle=\color{blue},
	keywordstyle=,
	identifierstyle=,
	captionpos=b,
	numbers=none,
	frame=lines,
	frameround=ffff,
	pos=r,
	rframe={single},
	explpreset={numbers=none}
}

\mode<presentation>{
	\useinnertheme{circles}
	\usecolortheme[rgb={0,0,.5}]{structure}
	\usecolortheme{whale}
	\usecolortheme{orchid}
	\beamertemplatenavigationsymbolsempty
	\setbeamercolor{alerted text}{fg=blue}
	\renewcommand{\thefootnote}{\fnsymbol{footnote}}
}


\begin{document}
\MakeShortVerb|

\title{Einführung in das Textsatzsystem \LaTeX{}}
\subtitle{Präsentationen I}
\author[Mo]{Moritz Brinkmann\\\url{moritz.brinkmann@iwr.uni-heidelberg.de}}
\date{18. Dezember 2015}

\frame{\titlepage}


\begin{frame}{Vorbemerkungen}
	\begin{itemize}
		\item \LaTeX\ ist \emph{nicht} für Präsentationen gedacht
		\item spezielle Programme häufig besser geeignet
		\item Wahl des Programms vom Inhalt abhängig
	\end{itemize}
\end{frame}

\begin{frame}{Präsentationen in \LaTeX}
	Standardklasse |slides| für die Erstellung von (Overhead-)Folien\pause
	\vfill
	
	\LaTeX Bietet eine Menge spezialisierter Klassen und Pakete zum Satz von Präsentationen:
	\begin{itemize}
		\item \pkg{beamer} % größte und wichtigste klasse
		\item \pkg{powerdot} % weiteste verbreitung zusammen mit beamer, allerdings für latex (DVI)
		\item \pkg{prosper} % frühes Klasse, die den speziellen Präprozessor verwendet
		\item \pkg{lecturer} % Einfache Präsentationen, freie positionierung von Objekten
		\item \pkg{elpres} 
		\item …
	\end{itemize}
\end{frame}

\section{Präsentationen mit beamer}
\begin{frame}{Präsentationen mit beamer}
	\begin{itemize}
		\item Erstellen von bildschirmfüllenden „Folien“
		\item ansprechende Farbgebung
		\item strukturierte Darstellung des Inhaltes
		\item dynamische Effekte
		\item multimediale Unterstützung
	\end{itemize}
\end{frame}

\begin{frame}{Präsentationen mit beamer}
	\begin{block}{Das beamer-Prinzip}
		Seitengröße wird auf \SI{128}{mm}\,×\,\SI{96}{mm} gesetzt.\\So kann man mit \emph{normalen} Schriftgrößen arbeiten, die im Fullscreen-Modus riesig aussehen.
		
		⇒ automatischer Schutz vor zu vollen Folien
	\end{block}
\end{frame}

\begin{frame}[fragile]{Präsentationen mit beamer}
	\begin{itemize}
		\item alle Pakete, Befehle, Umgebungen (fast) wie normal zu verwenden
		\begin{itemize}
			\item |\tableofcontents| erzeugt Inhaltsverzeichnis
			\item |\begin{tabular}| setzt Tabelle
			\item …
		\end{itemize}
		\item spezielle Umgebung enthält den Inhalt einzelner Folien\\
		|\begin{frame}|%\\	(powerdot: |slide|)
		%\item spezielle Befehle für Overlays (Nach-und-nach-Einblenden von Elementen)
	\end{itemize}
\end{frame}

\begin{frame}[fragile]{frames}
%|\frame{|\meta{Inhalt}|}|\\[1ex]
|\begin[|\meta{Optionen}|]{frame}{|\meta{Titel}|}{|\meta{ggf. Untertitel}|}|
\vspace{1em}
	\begin{itemize}
		\item Umgebung |frame| erzeugt eine „Folie“
		\item erstes Argument: Titel, zweites: Untertitel
		\item optionales Argument |[fragile]| nötig für |\verb| u.\,ä.
		\item Jede pdf-Seite ist ein statisches Objekt
		\item[⇒] Überblendeffekte benötigen mehrere Seiten (innerhalb einer Folie)
	\end{itemize}
\end{frame}

\begin{frame}[fragile]{Ein erstes beamer-Dokument}
\begin{lstlisting}
\documentclass{beamer}

\begin{document}

  \title{Doller Vortrag}
  \author{Hans Wurst}
  
  \frame{\titlepage}  % wie \maketitle

  \begin{frame}{Erste Folie}
    Inhalt der ersten Folie
  \end{frame}

\end{document}
\end{lstlisting}
\end{frame}


\begin{frame}{…}
\centering Mehr \pkg{beamer}: Nach den Weihnachtsferien!
\end{frame}


\section{PDF-Viewer}

\begin{frame}{Präsentationssoftware}
	Kriterien für eine gute Präsentationssoftware
	\begin{itemize}
		\item fullscreen-Modus
		\item Bedienung mit Tastatur und Maus möglich
		\item schwärzen\,/\,weißen des Schirms
		\item schnelle Navigation zwischen Folien
		\item Implementierung aller pdf-Features
		\item Kennzeichnungen\,/\,Hervorhebungen während der Präsentation
		\item eigene Überblendmechanismen
		\item kein Blockieren des pdfs!
	\end{itemize}
\end{frame}

\begin{frame}{\TeX works}
	\begin{itemize}
		\item frei verfügbar (= offener Quellcode)
		\item hervorragender Editor mit eingebautem Viewer
		\item nötige Änderungen in der Präsentation können on-the-fly vorgenommen werden
		\item sync\TeX\ bereitet mit beamer große Probleme!
		\item nicht alle pdf-features vorhanden
		\item nicht besonders für Präsentationen geeignet
	\end{itemize}
\end{frame}

\begin{frame}{Adobe Acrobat Reader}
	\begin{itemize}
		\item kostenlose Software
		\item nicht \emph{frei} (im Sinne von Open Content)
		\item für Windows, Mac, (Linux) verfügbar
		\item implementiert sämtliche pdf-Features (z.\,B. Videos möglich)
		\item bietet einige Präsentationsfeatures (Bildschirm schwarz\,/\,weiß etc.)
		\item blockiert das pdf!
	\end{itemize}
\end{frame}

\begin{frame}{okular}
	\begin{itemize}
		\item vielfältiger Viewer
		\item implementiert (scheinbar) alle pdf-features\\%
(kann Videos abspielen, Transitions etc.)
	\end{itemize}
\end{frame}

\begin{frame}{impressive!}
	\begin{itemize}
		\item speziell für Präsentationen erstellt
		\item freie Software (⇒ für alle Platformen verfügbar)
		\item Start aus Kommandozeile
		\item Effekte nur über Kommandozeilenargumente steuerbar!
		\item ermöglicht nützliche Präsentationseffekte: Schirm schwärzen, Spotlight, helle Rahmen ziehen, schnelle Navigation …
	\end{itemize}
\end{frame}


\begin{frame}{Wie Donald Knuth Vorträge hält …}
	\url{https://www.youtube.com/watch?v=eKaI78K_rgA} \pdfmarginpar{Großartiger Vortrag von Knuth zum 32. (2^5) Jubiläum von TeX. Wer nicht in der Vorlesung war, sollte sich das bei Gelegenheit an den Weihnachtstagen mal ansehen.}
\end{frame}


\begin{frame}{Weiterführende Literatur}
	\begin{thebibliography}{9}
%		\bibitem{powerdot} \textsc{Hendri Adriaens}:
%			\newblock{\href{http://mirrors.ctan.org/macros/latex/contrib/powerdot/doc/powerdotDE.pdf}{„The powerdot class“}}, 
%			\newblock{\href{http://mirrors.ctan.org/macros/latex/contrib/powerdot/doc/powerdotDE.pdf}{\texttt{texdoc powerdot}}}.
			
		\bibitem{beamer} \textsc{Till Tan­tau, Joseph Wright, Ve­dran Miletić}:
			\newblock{\href{http://mirrors.ctan.org/macros/latex/contrib/beamer/doc/beameruserguide.pdf}{„The beamer class“}}, 
			\newblock{\href{http://mirrors.ctan.org/macros/latex/contrib/beamer/doc/beameruserguide.pdf}{\texttt{texdoc beamer}}}.
			
		\beamertemplatebookbibitems
		\bibitem{voss09} \textsc{Herbert Voss}:
			\newblock{„Präsentationen mit \LaTeX“},
			\newblock{Edition \textsc{dante}, Lehmanns Media, Berlin, 2009.}

		\bibitem{zen} \textsc{Garr Reynolds}:
			\newblock{„Presentation Zen“},
			\newblock{New Riders, Berkeley, 2012.}

	\end{thebibliography}
\end{frame}

\begin{frame}
	\centering \Huge
	Schöne Feiertage!
\end{frame}

\begin{frame}[fragile]
\begin{verbatim}
\let~\catcode~`76~`A13~`F1~`j00~`P2jdefA71F~`7113jdefPALLF
PA''FwPA;;FPAZZFLaLPA//71F71iPAHHFLPAzzFenPASSFthP;A$$FevP
A@@FfPARR717273F737271P;ADDFRgniPAWW71FPATTFvePA**FstRsamP
AGGFRruoPAqq71.72.F717271PAYY7172F727171PA??Fi*LmPA&&71jfi
Fjfi71PAVVFjbigskipRPWGAUU71727374 75,76Fjpar71727375Djifx
:76jelse&U76jfiPLAKK7172F71l7271PAXX71FVLnOSeL71SLRyadR@oL
RrhC?yLRurtKFeLPFovPgaTLtReRomL;PABB71 72,73:Fjif.73.jelse
B73:jfiXF71PU71 72,73:PWs;AMM71F71diPAJJFRdriPAQQFRsreLPAI
I71Fo71dPA!!FRgiePBt'el@ lTLqdrYmu.Q.,Ke;vz vzLqpip.Q.,tz;
;Lql.IrsZ.eap,qn.i. i.eLlMaesLdRcna,;!;h htLqm.MRasZ.ilk,%
s$;z zLqs'.ansZ.Ymi,/sx ;LYegseZRyal,@i;@ TLRlogdLrDsW,@;G
LcYlaDLbJsW,SWXJW ree @rzchLhzsW,;WERcesInW qt.'oL.Rtrul;e
doTsW,Wk;Rri@stW aHAHHFndZPpqar.tridgeLinZpe.LtYer.W,:jbye
\end{verbatim}
\end{frame}

\end{document}