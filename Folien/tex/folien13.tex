% !TEX program = lualatex
% !TEX spellcheck = de_DE

\documentclass{beamer}

\newif\ifweb
\webfalse
\webtrue

\usepackage{
	dtklogos,
	fontspec,
	graphicx,
	hologo,
	mathtools,
	microtype,
	pdfmarginpar,
	polyglossia,
	qrcode,
	shortvrb,
	showexpl,
	siunitx,
	tikz,
	pgfplots,
	xspace,
}

\usepackage{
	booktabs,
	blindtext,
%	simpsons,
}

\setmainlanguage{german}
\setotherlanguage{english}

\setsansfont{Linux Biolinum O}
\setromanfont{Linux Libertine O}
\setmonofont[Scale=.95,AutoFakeSlant]{Inconsolata}
%\setmonofont[Scale=.99]{Anonymous Pro}


%% fancy-quotes %%%%%%%%%%%%%%
	\usepackage{tikz}
	\newcommand*\openquote{\makebox(25,-22){\scalebox{7}{\fontspec{Linux Biolinum O}“}}}
	\newcommand*\closequote{\makebox(25,-22){\scalebox{7}{\fontspec{Linux Biolinum O}”}}}
	\newcommand*{\OpenQuote}{\tikz[remember picture,overlay,xshift=-15pt,yshift=-10pt]
	     \node (OQ) {\openquote};}
	\newcommand*{\CloseQuote}{\tikz[remember picture,overlay,xshift=15pt,yshift=10pt]
	     \node (CQ) {\closequote};}
	\newenvironment{fancyquote}%
	{\hspace{-1em}\begin{quote}\OpenQuote\vspace*{1ex}\\\hspace*{1em}\begin{minipage}{.835\textwidth}}
	{\end{minipage}\vspace*{3.8ex}\\\hfill\CloseQuote\end{quote}}
	\newcommand\quoted[1]{\null\hfill{\scriptsize\sf #1}}
%%%%%%%%%%%%%%%%%%%%%%%%%%%%%%

\ifweb\else\renewcommand{\pdfmarginpar}[2][]{\null}\fi

%% overleaf %%%%%%%%%%%%%%%%%%
\usetikzlibrary{calc}
\tikzset{ href node/.style={alias=sourcenode,append after command={let \p1 = (sourcenode.north west),  \p2=(sourcenode.south east),\n1={\x2-\x1},\n2={\y1-\y2} in node [inner sep=0pt, outer sep=0pt,anchor=north west,at=(\p1)] {\href{#1}{\phantom{\rule{\n1}{\n2}}}}}}} % http://tex.stackexchange.com/a/36111
\newcommand{\overleaf}[1]{
	\begin{tikzpicture}[remember picture,overlay]
		\node [xshift=-1.2cm,yshift=1.5cm, href node={http://polr.me/#1}] at (current page.south east)
		{
			\scalebox{.55}{\parbox{4.1cm}{
				In Overleaf ausprobieren:\\[.8ex]
				\qrcode[height=4cm]{http://polr.me/#1}\\[1ex]
				\url{http://polr.me/#1}
			}}
		};
	\end{tikzpicture}
}
%%%%%%%%%%%%%%%%%%%%%%%%%%%%%%
\usetikzlibrary{shapes, arrows, trees}

\pgfplotsset{
	compat=1.12,
	width=7cm,
	lua backend=true,
}

\tikzset{onslide/.code args={<#1>#2}{%
  \only<#1>{\pgfkeysalso{#2}} % \pgfkeysalso doesn't change the path
}}
	
\newcommand{\meta}[1]{\textcolor{gray}{$\langle$\texttt{\textsl{#1}}$\rangle$}}
\newcommand{\pkg}[1]{\href{http://ctan.org/pkg/#1}{\alert{\texttt{#1}}}}
\newcommand{\TikZ}{Ti\textit{k}Z}
\newcommand{\BibTeX}{Bib\TeX}


\newenvironment{olcol}{
	\begin{columns}\begin{column}{.85\textwidth}
}{
	\end{column}\begin{column}{.108\textwidth}\relax\end{column}\end{columns}
}

\hypersetup{%
  unicode=true,
  pdfborder={000},
  pdftitle={Einführung in das Textsatzsystem LaTeX},
  pdfauthor={Moritz Brinkmann with credit to Arno Trautmann},
}

\lstloadlanguages{TeX}
\lstset{%
	language=TeX,
	backgroundcolor=\color[RGB]{229, 229, 239},
	basicstyle=\ttfamily\small,
	breakindent=0em,
	breaklines=true,
	commentstyle=\color{blue},
	keywordstyle=,
	identifierstyle=,
	captionpos=b,
	numbers=none,
	frame=lines,
	frameround=ffff,
	pos=r,
	rframe={single},
	explpreset={numbers=none}
}

\mode<presentation>{
	\useinnertheme{circles}
	\usecolortheme[rgb={0,0,.5}]{structure}
	\usecolortheme{whale}
	\usecolortheme{orchid}
	\beamertemplatenavigationsymbolsempty
	\setbeamercolor{alerted text}{fg=blue}
	\renewcommand{\thefootnote}{\fnsymbol{footnote}}
}


\begin{document}
\MakeShortVerb|

\title{Einführung in das Textsatzsystem \LaTeX{}}
\subtitle{komplexe Makros und Befehle}
\author[Mo]{Moritz Brinkmann\\\url{moritz.brinkmann@iwr.uni-heidelberg.de}}
\date{29. Januar 2016}

\frame{\titlepage}

\begin{frame}{Übersicht}
	\tableofcontents
\end{frame}

\section{Verschiedenes}
\subsection{Poster}
\begin{frame}{Poster}
∃ diverse Klassen für Satz von (wissenschaftlichen) Postern: \pkg{a0poster}, \pkg{sciposter}, \pkg{tikzposter}\\[1ex]\pause

Empfehlung: \pkg{tikzposter}\\[1ex]

Nutzt	 \TikZ\ um Objekte (Blocks, etc.) auf Poster zu platzieren. Bedienung vergleichar mit \pkg{beamer}.
\overleaf{tex1201}
\end{frame}

\subsection{Vorlesungsmitschriften}

\begin{frame}{Mitschreiben}
\begin{itemize}
	\item in Vorlesungen oder Übungen mit\TeX{}en manchmal nützlich
	\item entweder extrem hohe Tippgeschwindigkeit nötig
	\item oder durchdachte Befehlsdefinitionen
	\item \emph{wichtig:} alle strukturelle Information muss vorhanden sein!\\%
(auch, wenn es nicht gut aussieht)
\end{itemize} 
\end{frame}

\begin{frame}[fragile]{Mitschreiben}
\begin{itemize}
	\item häufig nur stichpunktartiges Aufschreiben
	\item[⇒] |\obeylines|
	\item Aufzählungen abkürzen, z.\,B. mittels |\let|\textbullet|\item| % damit ist „•“ gemeint
	\item …
\end{itemize}
\end{frame}


\section{Makros in \LaTeXe}
\subsection{newcommand, newenvironment \& Co}
\begin{frame}[fragile]{Makros in \LaTeX}
Zur Definition eigener Befehle in \LaTeX\ verfügbar:\\
|\newcommand|, |\renewcommand|, |\newenvironment|\\[2ex]
\pause
|\|(|re|)|newcommand{|\meta{Befehlsname}|}|\\
|  [|\meta{Anzahl der Argumente}|]|\\
|  [|\meta{Default für erstes (optionales) Argument}|]|\\
|  {|\meta{Befehlsdefinition}|}|\\[2ex]
|\newenvironment{|\meta{Umgebungsname}|}|\\
|  [|\meta{Anzahl der Argumente}|]|\\
|  [|\meta{Default für erstes (optionales) Argument}|]|\\
|  {|\meta{Definition Code \emph{vor} Umgebung}|}|\\
|  {|\meta{Definition Code \emph{nach} Umgebung}|}|\\[2ex]

\pause
Varianten mit Stern: |\newcommand*| für zusätzliche Fehler-Checks, falls Argumente \emph{keine} Umbrüche/Leerzeilen enthalten dürfen
\end{frame}

\subsection{def und let}
\begin{frame}[fragile]{Makros in \TeX}
\TeX\ bietet die Primitiven |\def| und |\let|\\[2ex]\pause

|\def|\meta{Befehlsname}\meta{Argument(e)}|{|\meta{Befehlsdefinition}|}|\\
\begin{lstlisting}
\def\mymakro#1#2{Makro mit zwei Argumenten #1 und #2}
\end{lstlisting}\pause
\vspace*{2ex}

|\let|\meta{neuer Befehlsname}\meta{alter Befehlsname}\\
\begin{lstlisting}
\let\newmakro\oldmakro
\end{lstlisting}
\begin{itemize}
\item generiert |\newmakro| mit exakt den selben Eigenschaften wie |\oldmakro|
\item wenn sich |\oldmakro| ändert, bleibt |\newmakro| erhalten
\end{itemize}
\end{frame}


\begin{frame}[fragile]{Makros in \TeX}
\begin{itemize}
\item |\def| und |\let| auch in \LaTeX\ verfügbar
\item High-Level Befehle wie |\newcommand| sind meist vorzuziehen
\item |\let| manchmal praktisch
\item  nur benutzen, wenn man weiß was man tut
\end{itemize}
\end{frame}


\subsection{Naming Conventions}
\begin{frame}[fragile]{Naming Conventions}
\begin{description}
\item[|lowercase|] Endnutzer-Befehle auf Dokumenten-Level\\(braucht man ständig)
\item[|MixedCase|] Befehle für spezielle Funktionen in Paketen oder Klassen\\(braucht man selten)
\item[|with@sign|] interne Befehle in Paketen oder im \LaTeX-Kernel\\(braucht man \emph{nie})
\end{description}
\vfill\pause
\begin{block}{spezieller Schutzmechanismus}
@-Zeichen hat anderen \textenglish{category code} als normale Buchstaben, Befehle mit @ werden daher ignoriert.\\
Ausschalten: |\makeatletter|\\
wieder Einschalten: |\makeatother|
\end{block}
\vfill\pause
\TeX-Primitiven sind – aus historischen Gründen – auch lowercase
\end{frame}


\section{\LaTeX3}
\begin{frame}{Makros in \LaTeX3}
	\begin{columns}\hspace{-2.5em}
		\begin{column}{.7\textwidth}
			\begin{itemize}
				\item Mit \LaTeX3 wird alles besser:
				\begin{itemize}
					\item Konsequente Unterscheidung zwischen Nutzer-, Design- und Programmierebene
					\item Namespaces für Pakete
					\item sehr bequeme und flexible Befehlsdefinitionen
				\end{itemize}
				\item \LaTeX3-Syntax schon jetzt nutzbar:
				\begin{itemize}
					\item Paket \pkg{expl3} für Entwickler
					\item Paket \pkg{xparse} für Endnutzer
				\end{itemize}
			\end{itemize}
		\end{column}
		\begin{column}{.2\textwidth}
			\vspace{-2cm}
			\includegraphics[width=1.5\textwidth]{expl3}
			\overleaf{tex130}
		\end{column}
	\end{columns}
\end{frame}


\subsection{Makros in \LaTeX3}
\begin{frame}[fragile]{Makros in \LaTeX3}%{xparse}

Mit Paket \pkg{xparse} verfügbar:\\
|\NewDocumentCommand|, |\RenewDocumentCommand|, |\NewDocumentEnvironment|, …
\vfill

|\NewDocumentCommand{|\meta{Befehlsname}|}|\\|  {|\meta{Argumentstruktur}|}|\\|  {|\meta{Definition}|}|
\vfill

 \meta{Argumentstruktur} beschreibt wie viele und welche Argumente\\der Befehl annimmt (sozusagen die Signatur)
\end{frame}


\begin{frame}[fragile]{Argumentstruktur}
\begin{block}<1>{mandatorische Argumente}
\begin{description}
\item[|m|] klassisches mandatorisches Argument \hfill |{|\meta{...}|}|
\item[|l|] liest alles vor der nächsten Klammer \hfill \meta{...}|{|
\item[|r|\meta{t1}\meta{t2}] alles zwischen \meta{t1} und \meta{t2}\quad
	z.\,B. |r<>| \hfill |<|\meta{...}|>|
%\item[|R|\meta{token1} und \meta{token2}|{|\meta{Default}|}|] wie |r| nur mit default-Wert
\item[|u\{|\meta{t}|\}|] liest alles bis \meta{t}\quad
	z.\,B. |u{§&}| \hfill \meta{...}|§&|
\item[|v|] Verbatim-Input \hfill \texttt{\textbar\meta{...}\textbar}\\
	Eingabe wird nicht interpretiert \hfill \texttt{\{\meta{...}\}}\\
\end{description}
\end{block}
\vfill
\begin{lstlisting}
\NewDocumentCommand{\mycommand}{ m l m r^° }
  {(#1,#2,#3,#4)}
\mycommand{eins}zwei{drei}^vier°
\end{lstlisting}
\end{frame}

\begin{frame}[fragile]{Argumentstruktur}
\begin{block}{optionale Argumente}
\begin{description}
\item[|o|] klassisches optionales Argument \hfill |[|\meta{...}|]|
\item[|O\{|\meta{df}|\}|] wie |o| mit Default-Wert \quad z.\,B. |O{default}| \hfill |[|\meta{...}|]|
\item[|d|\meta{t1}\meta{t2}] alles zwischen \meta{t1} und \meta{t2}\quad
	z.\,B. |d:+| \hfill |:|\meta{...}|+|
\item[|D|\meta{t1}\meta{t2}|\{|\meta{df}|\}|] wie |d| mit Default-Wert \quad
	z.\,B. |d(){bla}| \hfill |(|\meta{...}|)|
\item[|s|] Stern, setzt |\BooleanTrue|, falls vorhanden \hfill |*|
\end{description}
\end{block}
\vfill
\begin{lstlisting}
\NewDocumentCommand{\mycommand}
  { d<| O{zwei} s D|>{vier} }
  { (#1,#2,#4) \IfBooleanT{#3}{:-)} }
\mycommand<eins|[2]*
\end{lstlisting}
\end{frame}

\begin{frame}[fragile]{Argumentstruktur}
\begin{block}{Argument-Modifier}
\begin{description}
\item[|+|\meta{Arg-Kürzel}] erlaubt Eingabe von Umbrüchen innerhalb eines Arguments \\
	z.\,B. |+m|
\item[|>\{|\meta{Prozessor}|\}|] Argumente vor dem Auslesen bearbeiten \\
	z.\,B. |> {\ReverseBoolean} m|\\|> {\TrimSpaces} o|
\end{description}
\end{block}
\hfill
\begin{lstlisting}
\NewDocumentCommand{\mycommand}
  { >{\ReverseBoolean} s o +m }
  { \IfBooleanTF{#1}{kein stern}{stern} #2 #3 }
\mycommand*{Text mit\\Umbruch}
\end{lstlisting}
\end{frame}

\begin{frame}[fragile]{Umgebungen}
|\NewDocumentEnvironment{|\meta{Umgebungsname}|}|\\
|  {|\meta{Argumentstruktur}|}|\\
|  {|\meta{Definition Code \emph{vor} Umgebung}|}|\\
|  {|\meta{Definition Code \emph{nach} Umgebung}|}|\vfill

\begin{lstlisting}
\newDocumentEnvironment{myquote}{ o }
  {\begin{quote}\sffamily\itshape}
  {\end{quote}\IfNoValueF{#1}{Quelle:#1}}
  
\begin{myquote}[Internet]
  Bla bla, Chemtrails, Lügenpresse …
\end{myquote}
\end{lstlisting}
\end{frame}


\begin{frame}[fragile]{expl3}
Erweiterte \LaTeX3-Funktionalität für Entwickler mit \pkg{expl3} verfügbar\\[1ex]

|\ExplSyntaxOn|\\
|  |\meta{Code}\\
|\ExplSyntaxOff|\\[1ex]

Schaltet neue Syntax ein und aus
\end{frame}

\section{\hologo{LuaLaTeX}}
\begin{frame}[fragile]{Nutzung von Lua mit \hologo{LuaLaTeX}}
Innerhalb von \TeX\ Lua-Code eingeben:\\
|\directlua{|\meta{Lua-Code}|}|\\[1ex]

Innerhalb von Lua-Code etwas an \TeX\ ausgeben:\\
|tex.print(|\meta{TeX-Ausgabe}|)|\\[3ex]
\pause

\begin{LTXexample}
$\pi = \directlua{
  tex.sprint(math.pi)
}$
\end{LTXexample}

\overleaf{tex1303}
\end{frame}

\begin{frame}[fragile]{Nutzung von Lua mit \hologo{LuaLaTeX}}
\begin{itemize}
	\item |\directlua| macht manchmal Probleme
	\begin{itemize}
		\item bei Umbrüchen
		\item bei Lua-Kommentaren |---|
		\item bei Sonderzeichen |_^&${}%|
	\end{itemize}
	\item Paket \pkg{luacode} behebt viele dieser Probleme.:\\[1ex]
|\begin{luacode*}|\\
|  |\meta{Lua-Code}\\
|\end{luacode*}|

	\item in Variante mit Stern sind keine \TeX-Befehle möglich
	\item in Variante ohne Stern werden \TeX-Makros expandiert
\end{itemize}

\overleaf{tex1303}
\end{frame}


\begin{frame}{Weiterführende Literatur}
	\begin{thebibliography}{9}
		\bibitem{xparse} \textsc{The \LaTeX3 Project}:
			\newblock{\href{http://mirrors.ctan.org/macros/latex/contrib/l3packages/xparse.pdf}{„The |xparse| package Document command parser“}}, 
			\newblock{\href{http://mirrors.ctan.org/macros/latex/contrib/l3packages/xparse.pdf}{\texttt{texdoc xparse}}}.
			
		\bibitem{interface3} \textsc{The \LaTeX3 Project}:
			\newblock{\href{http://mirrors.ctan.org/macros/latex/contrib/l3kernel/interface3.pdf}{„The \LaTeX3 in­ter­faces“}},
\newblock{\href{http://mirrors.ctan.org/macros/latex/contrib/l3kernel/interface3.pdf}{\texttt{texdoc interface3}}}.
			
		\bibitem{expl3} \textsc{The \LaTeX3 Project}:
			\newblock{\href{http://mirrors.ctan.org/macros/latex/contrib/l3kernel/expl3.pdf}{„The |expl3| package and \LaTeX3 programming“}}, 
			\newblock{\href{http://mirrors.ctan.org/macros/latex/contrib/l3kernel/expl3.pdf}{\texttt{texdoc expl3}}}.
			
		\bibitem{luatex-doc} \textsc{Manuel Pégourié-Gonnard}:
			\newblock{\href{http://mirrors.ctan.org/macros/latex/contrib/lualatex/lualatex-doc.pdf}{„A Guide to \hologo{LuaLaTeX}“}}, 
			\newblock{\href{http://mirrors.ctan.org/macros/latex/contrib/lualatex/lualatex-doc.pdf}{\texttt{texdoc lualatex}}}.
			
		\bibitem{luacode} \textsc{Manuel Pégourié-Gonnard}:
			\newblock{\href{http://mirrors.ctan.org/macros/latex/contrib/lualatex/luacode.pdf}{„The luacode package“}}, 
			\newblock{\href{http://mirrors.ctan.org/macros/latex/contrib/lualatex/luacode.pdf}{\texttt{texdoc luacode}}}.

	\end{thebibliography}
\end{frame}


\begin{frame}{\TeX-Stammtisch}
\begin{itemize}
	\item In vielen Städten gibt es aktive \TeX-Nutzergruppen
	\item Hier: Heidelberger \TeX-Stammtisch (zu Ladenburg)
	\begin{itemize}
		\item lustige Runde, es geht nicht immer um \TeX
		\item jeden letzten Freitag im Monat
		\item in wechselnden Restaurants
		\item Ankündigung auf \href{http://tinyurl.com/Stammtisch-HD}{Mailingliste}
	\end{itemize}
\end{itemize}
\pause \vfill
\begin{block}{Nächster Stammtisch: Heute!}
	29.01.2016, 19:30 Uhr\\
	Restaurant zum Löwen, Mühltalstraße 1, HD-Handschuhsheim
\end{block}
\end{frame}

\end{document}