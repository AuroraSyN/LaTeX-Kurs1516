% !TEX program = xelatex
% !TEX spellcheck = de_DE

\documentclass{beamer}

\newif\ifweb
\webfalse
\webtrue

\usepackage{
	dtklogos,
	fontspec,
	graphicx,
	hologo,
	mathtools,
	microtype,
	pdfmarginpar,
	polyglossia,
	qrcode,
	shortvrb,
	showexpl,
	siunitx,
	tikz,
	pgfplots,
	xspace,
}

\usepackage{
	booktabs,
	blindtext,
%	simpsons,
}

\setmainlanguage{german}
\setotherlanguage{english}


\usepackage{comicneue,skull,CountriesOfEurope}
\newfont{\cirth}{cirth scaled\magstep1}
%\newfont{\ransom}{ransom10 scaled\magstep1}  % leider fehlen dazu die map-files
\newfont{\dancers}{dancers scaled\magstep1}

\setsansfont{Linux Biolinum O}
\setromanfont{Linux Libertine O}
\setmonofont[Scale=.95,AutoFakeSlant]{Inconsolata}
%\setmonofont[Scale=.99]{Anonymous Pro}


%% fancy-quotes %%%%%%%%%%%%%%
	\usepackage{tikz}
	\newcommand*\openquote{\makebox(25,-22){\scalebox{7}{\fontspec{Linux Biolinum O}“}}}
	\newcommand*\closequote{\makebox(25,-22){\scalebox{7}{\fontspec{Linux Biolinum O}”}}}
	\newcommand*{\OpenQuote}{\tikz[remember picture,overlay,xshift=-15pt,yshift=-10pt]
	     \node (OQ) {\openquote};}
	\newcommand*{\CloseQuote}{\tikz[remember picture,overlay,xshift=15pt,yshift=10pt]
	     \node (CQ) {\closequote};}
	\newenvironment{fancyquote}%
	{\hspace{-1em}\begin{quote}\OpenQuote\vspace*{1ex}\\\hspace*{1em}\begin{minipage}{.835\textwidth}}
	{\end{minipage}\vspace*{3.8ex}\\\hfill\CloseQuote\end{quote}}
	\newcommand\quoted[1]{\null\hfill{\scriptsize\sf #1}}
%%%%%%%%%%%%%%%%%%%%%%%%%%%%%%

\ifweb\else\renewcommand{\pdfmarginpar}[2][]{\null}\fi

%% overleaf %%%%%%%%%%%%%%%%%%
\usetikzlibrary{calc}
\tikzset{ href node/.style={alias=sourcenode,append after command={let \p1 = (sourcenode.north west),  \p2=(sourcenode.south east),\n1={\x2-\x1},\n2={\y1-\y2} in node [inner sep=0pt, outer sep=0pt,anchor=north west,at=(\p1)] {\href{#1}{\phantom{\rule{\n1}{\n2}}}}}}} % http://tex.stackexchange.com/a/36111
\newcommand{\overleaf}[1]{
	\begin{tikzpicture}[remember picture,overlay]
		\node [xshift=-1.2cm,yshift=1.5cm, href node={http://polr.me/#1}] at (current page.south east)
		{
			\scalebox{.55}{\parbox{4.1cm}{
				In Overleaf ausprobieren:\\[.8ex]
				\qrcode[height=4cm]{http://polr.me/#1}\\[1ex]
				\url{http://polr.me/#1}
			}}
		};
	\end{tikzpicture}
}
%%%%%%%%%%%%%%%%%%%%%%%%%%%%%%
\usetikzlibrary{shapes, arrows, trees}

\pgfplotsset{
	compat=1.12,
	width=7cm,
	lua backend=true,
}

\tikzset{onslide/.code args={<#1>#2}{%
  \only<#1>{\pgfkeysalso{#2}} % \pgfkeysalso doesn't change the path
}}
	
\newcommand{\meta}[1]{\textcolor{gray}{$\langle$\texttt{\textsl{#1}}$\rangle$}}
\newcommand{\pkg}[1]{\href{http://ctan.org/pkg/#1}{\alert{\texttt{#1}}}}
\newcommand{\TikZ}{Ti\textit{k}Z}
\newcommand{\BibTeX}{Bib\TeX}


\newenvironment{olcol}{
	\begin{columns}\begin{column}{.8\textwidth}
}{
	\end{column}\begin{column}{.108\textwidth}\relax\end{column}\end{columns}
}

\hypersetup{%
  unicode=true,
  pdfborder={000},
  pdftitle={Einführung in das Textsatzsystem LaTeX},
  pdfauthor={Moritz Brinkmann with credit to Arno Trautmann},
}

\lstloadlanguages{TeX}
\lstset{%
	language=TeX,
	backgroundcolor=\color[RGB]{229, 229, 239},
	basicstyle=\ttfamily\small,
	breakindent=0em,
	breaklines=true,
	commentstyle=\color{blue},
	keywordstyle=,
	identifierstyle=,
	captionpos=b,
	numbers=none,
	frame=lines,
	frameround=ffff,
	pos=r,
	rframe={single},
	explpreset={numbers=none}
}

\mode<presentation>{
	\useinnertheme{circles}
	\usecolortheme[rgb={0,0,.5}]{structure}
	\usecolortheme{whale}
	\usecolortheme{orchid}
	\beamertemplatenavigationsymbolsempty
	\setbeamercolor{alerted text}{fg=blue}
	\renewcommand{\thefootnote}{\fnsymbol{footnote}}
}


\begin{document}
\MakeShortVerb|

\title{Einführung in das Textsatzsystem \LaTeX{}}
\subtitle{Präsentationen II}
\author[Mo]{Moritz Brinkmann\\\url{moritz.brinkmann@iwr.uni-heidelberg.de}}
\date{18. Dezember 2015}

\frame{\titlepage}

\section{Präsentationen mit beamer}
\begin{frame}{Präsentationen mit beamer}
	\begin{itemize}
		\item Erstellen von bildschirmfüllenden „Folien“
		\item ansprechende Farbgebung
		\item strukturierte Darstellung des Inhaltes
		\item dynamische Effekte
		\item multimediale Unterstützung
	\end{itemize}
\end{frame}

\begin{frame}{Präsentationen mit beamer}
	\begin{block}{Das beamer-Prinzip}
		Seitengröße wird auf \SI{128}{mm}\,×\,\SI{96}{mm} gesetzt.\\So kann man mit \emph{normalen} Schriftgrößen arbeiten, die im Fullscreen-Modus riesig aussehen.
		
		⇒ automatischer Schutz vor zu vollen Folien
	\end{block}
\end{frame}

\begin{frame}[fragile]{Präsentationen mit beamer}
	\begin{itemize}
		\item alle Pakete, Befehle, Umgebungen (fast) wie normal zu verwenden
		\begin{itemize}
			\item |\tableofcontents| erzeugt Inhaltsverzeichnis
			\item |\begin{tabular}| setzt Tabelle
			\item …
		\end{itemize}
		\item spezielle Umgebung enthält den Inhalt einzelner Folien\\
		|\begin{frame}|%\\	(powerdot: |slide|)
		%\item spezielle Befehle für Overlays (Nach-und-nach-Einblenden von Elementen)
	\end{itemize}
\end{frame}

\begin{frame}[fragile]{frames}
%|\frame{|\meta{Inhalt}|}|\\[1ex]
|\begin{frame}[|\meta{Optionen}|]{|\meta{Titel}|}{|\meta{ggf. Untertitel}|}|
\vspace{1em}
	\begin{itemize}
		\item Umgebung |frame| erzeugt eine „Folie“
		\item erstes Argument: Titel, zweites: Untertitel
		\item optionales Argument |[fragile]| nötig für |\verb| u.\,ä.
		\item Jede pdf-Seite ist ein statisches Objekt
		\item[⇒] Überblendeffekte benötigen mehrere Seiten (innerhalb einer Folie)
	\end{itemize}
\end{frame}

\begin{frame}[fragile]{Ein erstes beamer-Dokument}
\begin{olcol}
\begin{lstlisting}
\documentclass{beamer}

\begin{document}

  \title{Doller Vortrag}
  \author{Hans Wurst}
  
  \frame{\titlepage}

  \begin{frame}{Erste Folie}
    Inhalt der ersten Folie
  \end{frame}

\end{document}
\end{lstlisting}
\end{olcol}
\overleaf{tex1101}
\end{frame}


\begin{frame}[fragile,t]{vertikale Ausrichtung}
\overleaf{tex1101}%
vertikale Ausrichtung mittels optionalem Argument |[t,b,c]|, auch als Dokumentklassenoption
\vspace{2ex}

|\begin{frame}[t]{|\meta{Titel}|}{|\meta{Untertitel}|}|\\
|  |\meta{Folieninhalt}\\
|\end{frame}|
\end{frame}

\begin{frame}{Überblendeffekte}
\overleaf{tex1101}
\begin{itemize}
    \item für dynamische Effekte: |<|\meta{Kürzel}|>|
    \item<+-> |<+->|\phantom{\texttt{5}} lässt Objekt erscheinen und bleiben
    \item<+> |<+>|\phantom{\texttt{-5}} lässt Objekt erscheinen und wieder verschwinden
    \item<4> |<4>|\phantom{\texttt{-5}} Objekt erscheint auf Folie 4
    \item<4-5> |<4-5>| Objekt erscheint auf Folien 4 bis 5  
    \item<5> |<0>| Objekt erscheint gar nicht
\end{itemize}
\end{frame}
    
\begin{frame}[fragile]{Überblendeffekte}
z.\,B. bei |itemize|:

\begin{lstlisting}
\begin{itemize}[<+->]  % Angabe gilt für alle \items
  \item<+-> Punkt 1
  \item<4> Punkt 2
  \item<+-> Punkt 3
\end{itemize}
\end{lstlisting}
Auch bei |\includegraphics<|\meta{Kürzel}|>| u.\,a.
\end{frame}

\begin{frame}[fragile]{Überblendeffekte}{Pause}
\begin{itemize}
    \item {|\pause|} stoppt den Inhalt an beliebiger Stelle
    \item erste Seite wird bis |\pause| gesetzt
    \pause
    \item zweite Seite enthält den gesamten Inhalt (bis zum nächsten |\pause|)
\end{itemize}
\[a =\pause b_{c\pause \cdot d}\]
\end{frame}

\begin{frame}[fragile]{Überblendeffekte}{only}
\begin{itemize}
    \item |\only<|\meta{Kürzel}|>{|\meta{Inhalt}|}| setzt den \meta{Inhalt} nur in den angegeben Seiten
    \item Platz für den \meta{Inhalt} wird \emph{nicht} freigehalten
    \item |\only<4>{|\meta{Inhalt}|}| setzt nur in der vierten Seite
    \item |\only<3->{|\meta{Inhalt}|}| setzt ab der dritten Seite
\end{itemize}
\end{frame}

\begin{frame}[fragile]{Überblendeffekte}{uncover}
\begin{itemize}
    \item|\uncover<|\meta{Kürzel}|>{|\meta{Inhalt}|}| setzt den \meta{Inhalt} nur in den angegeben Seiten
    \item Platz für den \meta{Inahlt} \emph{wird} freigehalten
    \item |\uncover<4>{|\meta{Inhalt}|}| setzt nur in der vierten Seite
    \item |\uncover<3->{|\meta{Inhalt}|}| setzt ab der dritten Seite
\end{itemize}
\end{frame}

\begin{frame}[fragile]{themes}{allgemeine}
\begin{itemize}
    \item themes sind Stilvorlagen, die das gesamte Layout beeinflussen
    \item Einbinden mittels |\usetheme| im Header
    \item benannt nach Tagungsorten
    \item siehe \pkg{beamer}-Dokumentation oder \url{https://hartwork.org/beamer-theme-matrix/}
\end{itemize}
\end{frame}

\begin{frame}[fragile]{themes}{inner}
\begin{itemize}
    \item beeinflussen das Aussehen von Elementen in der Folie
    \item Aufzählungen, Abbildungsbeschriftung, Boxen etc.
    \item |\useinnertheme|
\end{itemize} 
\end{frame}

\begin{frame}[fragile]{themes}{outer}
\begin{itemize}
    \item beeinflussen das Aussehen der äußeren Element
    \item Kopfzeile, Fußzeile, Navigation etc.
    \item |\useoutertheme|
\end{itemize} 
\end{frame}

\begin{frame}[fragile]{themes}{color}
\begin{itemize}
    \item wie der Name sagt …
    \item je nach Theme werden verschiedene Elemente coloriert
    \item Farben für jedes Element anpassbar:\\
\begin{lstlisting}
\setbeamercolor{footnote}{fg=red}
\end{lstlisting}
    \item |fg| für |foreground|, |bg| für |background|
\end{itemize}
\end{frame}

\begin{frame}[fragile]{themes}{font}
\begin{itemize}
    \item ändert Auswahl der Schriftarten
    \item |default| (Serifenlose), |serif|, |structurebold|, |structuresmallcapserif|, …\\
    |professionalfont| (für professionelle (gekaufte) Schriften)
\end{itemize}
\end{frame}

\begin{frame}[fragile]{Navigationselemente}
	\begin{center}
	\scalebox{2}{
		\insertslidenavigationsymbol \insertframenavigationsymbol \insertsubsectionnavigationsymbol \insertsectionnavigationsymbol \insertdocnavigationsymbol \insertbackfindforwardnavigationsymbol}
	\end{center}
	\begin{itemize}
		\item Erlauben Springen zwischen Folien, Frames, (Sub-)Sections, …
		\item normalerweise in der rechten unteren Ecke
		\item Ausblenden mit |\textbackslash beamertemplatenavigationsymbolsempty|
	\end{itemize}
\end{frame}

\begin{frame}[fragile]{Gliederung}
\begin{itemize}
    \item normale Gliederungselemente vorhanden
    \item |\section, \subsection, \chapter, ...|
    \item Angabe von |\section| bewirkt zunächst nichts!\\%
(Absatzüberschriften werden \emph{nicht} ausgegeben)
    \item Einfluss nur in Inhaltsverzeichnissen und Headern
\end{itemize}
\end{frame}

\begin{frame}[fragile]{Strukturelemente}{block}
\begin{LTXexample}[rframe={}]
\begin{block}{Titel}
Inhalt eines schön gefärbten Blockes.
\end{block}
\begin{block}<2>{Zwei}
Und noch einer.
\end{block}
\end{LTXexample}
\end{frame}

\begin{frame}[fragile]{Strukturelemente}{theorem}
\begin{LTXexample}[rframe={}]
\begin{theorem}[Trautmann et al.]
1 + 2 = 3
\end{theorem}
\begin{proof}
2 = 1+1\\
1+1+1 = 3
\end{proof}
\begin{example}
2+1 = 3
\end{example}
\end{LTXexample}
Konflikt mit |theorem| aus \pkg{amsmath}!\\
Können nummeriert werden mit Dokumentenoption |[envcountsec]|
\end{frame}

\section{Multimedia}
\begin{frame}[fragile]{Gleitumgebungen}
\begin{itemize}
    \item Einfügen von Abbildungen, Tabellen u.\,ä. wie gewohnt
    \item Gleitumgebungen werden nicht nummeriert
    \item Positionsangaben (h,t,b) werden ignoriert
    \item |\logo| fügt ein Logo global in die Präsentation ein (z.\,B. oben links)
    \item Bilder einfügen mittels |\includegraphics| oder:
\end{itemize}
\begin{lstlisting}
\pgfdeclareimage[height=0.5cm]{logo}{tu-logo}
\logo{\pgfuseimage{logo}}
\logo{\includegraphics[height=0.5cm]{logo}{tu-logo}}
\end{lstlisting}
\end{frame}

\begin{frame}{Filme}
\begin{itemize}
    \item Paket |multimedia| (gehört zu beamer) laden
    \item unter Verwendung von pdf\LaTeX\ und geeignetem Viewer: Einbinden von Videos möglich
\end{itemize}
\end{frame}

\iftrue
\begin{frame}[fragile]{Modi}
\begin{itemize}
	\item \pkg{beamer} kann mit verschiedenen Modi umgehen
	\item |presentation| (Standard), |handout|, |article|, …
	\item |handout| entfernt alle overlays
	\item |\only<|\meta{Modus}|>{|\meta{Inhalt}|}|
\end{itemize}
\begin{lstlisting}
\begin{frame}<handout:0> %versteckt ganze Folie
  \only<4|article:3>{Bla}
  ...
\end{lstlisting}
\end{frame}
\fi

% modi: \begin{frame}<0| handout:0>; handout entfernt overlays
% shrink, plain, allowframebreak
% \beamertemplatenavigationsymbolsempty
% Literatur: beamer (besonders abschnitt 5)


\begin{frame}{Weiterführende Literatur}
\begin{thebibliography}{9}
\bibitem{beamer} \textsc{Till Tantau, Joseph Wright, Vedran Miletić}:
	\newblock{\href{http://mirrors.ctan.org/macros/latex/contrib/beamer/doc/beameruserguide.pdf}{„The beamer class“}}
	\newblock{\href{http://mirrors.ctan.org/macros/latex/contrib/beamer/doc/beameruserguide.pdf}{\texttt{texdoc beamer}}}
	\newblock{\emph{Besonders lesenswert: Kapitel 5}}
\end{thebibliography}
\end{frame}


\end{document}