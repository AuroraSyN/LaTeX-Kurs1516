% !TEX TS-program = lualatex
% !TEX encoding = UTF-8 Unicode
% !TEX spellcheck = de_DE

\documentclass{beamer}

\newif\ifweb
\webfalse
\webtrue

\usepackage{
	fontspec,
	graphicx,
	microtype,
	pdfmarginpar,
	polyglossia,
	qrcode,
	shortvrb,
	showexpl,
	tikz,
}

\setmainlanguage{german}

\setsansfont{Linux Biolinum O}
\setromanfont{Linux Libertine O}
\setmonofont[Scale=.95,AutoFakeSlant]{Inconsolata}

%\renewcommand{\sfdefault}{\rmdefault}


%% fancy-quotes %%%%%%%%%%%%%%
	\usepackage{tikz}
	\newcommand*\openquote{\makebox(25,-22){\scalebox{7}{\fontspec{Linux Biolinum O}“}}}
	\newcommand*\closequote{\makebox(25,-22){\scalebox{7}{\fontspec{Linux Biolinum O}”}}}
	\newcommand*{\OpenQuote}{\tikz[remember picture,overlay,xshift=-15pt,yshift=-10pt]
	     \node (OQ) {\openquote};}
	\newcommand*{\CloseQuote}{\tikz[remember picture,overlay,xshift=15pt,yshift=10pt]
	     \node (CQ) {\closequote};}
	\newenvironment{fancyquote}%
	{\hspace{-1em}\begin{quote}\OpenQuote\vspace*{1ex}\\\hspace*{1em}\begin{minipage}{.835\textwidth}}
	{\end{minipage}\vspace*{3.8ex}\\\hfill\CloseQuote\end{quote}}
	\newcommand\quoted[1]{\null\hfill{\scriptsize\sf #1}}
%%%%%%%%%%%%%%%%%%%%%%%%%%%%%%

\ifweb\else\renewcommand{\pdfmarginpar}[2][]{\null}\fi

%% overleaf %%%%%%%%%%%%%%%%%%
\usetikzlibrary{calc}
\tikzset{ href node/.style={alias=sourcenode,append after command={let \p1 = (sourcenode.north west),  \p2=(sourcenode.south east),\n1={\x2-\x1},\n2={\y1-\y2} in node [inner sep=0pt, outer sep=0pt,anchor=north west,at=(\p1)] {\href{#1}{\phantom{\rule{\n1}{\n2}}}}}}} % http://tex.stackexchange.com/a/36111
\newcommand{\overleaf}[1]{
	\begin{tikzpicture}[remember picture,overlay]
		\node [xshift=-1.2cm,yshift=1.5cm, href node={http://polr.me/#1}] at (current page.south east)
		{
			\pdfmarginpar{Dieser Link oeffnet ein LaTeX-Dokument im Webdienst Overleaf, in dem die hier erklaerten Befehle gleich ausprobiert werden koennen.}
			\scalebox{.55}{\parbox{4.1cm}{
				In Overleaf ausprobieren:\\[.8ex]
				\qrcode[height=4cm]{http://polr.me/#1}\\[1ex]
				\url{http://polr.me/#1}
			}}
		};
	\end{tikzpicture}
}
%%%%%%%%%%%%%%%%%%%%%%%%%%%%%%

\newcommand{\meta}[1]{\textcolor{gray}{$\langle$\texttt{\textsl{#1}}$\rangle$}}

\hypersetup{%
  unicode=true,
  pdfborder={000},
  pdftitle={Einführung in das Textsatzsystem LaTeX},
  pdfauthor={Moritz Brinkmann with credit to Arno Trautmann},
}

\lstset{%
	backgroundcolor=\color[RGB]{229, 229, 239},
	basicstyle=\ttfamily\small,
	breakindent=0em,
	breaklines=true,
	commentstyle=\color{blue},
	keywordstyle=,
	identifierstyle=,
	captionpos=b,
	numbers=none,
	frame=lines,
	frameround=ffff,
	pos=r,
	rframe={single},
	explpreset={numbers=none}
}

\mode<presentation>{
	\useinnertheme{circles}
	\usecolortheme[rgb={0,0,.5}]{structure}
	\usecolortheme{whale}
	\usecolortheme{orchid}
	\beamertemplatenavigationsymbolsempty
	\setbeamercolor{alerted text}{fg=blue} 
}

\title{Einführung in das Textsatzsystem \LaTeX{}}
\subtitle{Einführung und grundlegende Bedienung}
\author[Mo]{Moritz Brinkmann\\\url{moritz.brinkmann@iwr.uni-heidelberg.de}}
\date{16. Oktober 2015}


\begin{document}
\MakeShortVerb{\|}

\frame{\titlepage}

\section{Organisatorisches}

\begin{frame}[t]{Organisatorisches}%{Allgemeines}
	\begin{block}<1->{Anmeldung}
		\begin{itemize}
			\item	 Anmeldung im \href{https://www.mathi.uni-heidelberg.de/muesli/lecture/view/484}{MÜSLI} obligatorisch
			\item Verwaltung der Zettelpunkte über \href{https://www.mathi.uni-heidelberg.de/muesli/lecture/view/484}{MÜSLI}
		\end{itemize}
	\end{block}
	\begin{block}<2>{Materialien}
		\begin{itemize}
			\item Übungszettel und Vorlesungsfolien stehen auf der \href{http://latexkurs.de}{Vorlesungshomepage} zum Download
		\end{itemize}
	\end{block}
\end{frame}

\begin{frame}[t]{Organisatorisches}%{Übungsbetrieb}
	\begin{block}{Prüfungsmodalitäten}
		\begin{itemize}
			\item Bearbeiten der wöchentlichen Übungszettel
			\item Erreichen von mindestens 50\% der möglichen Gesamtpunktzahl
			\item Übungspunkte ergeben Scheinnote 
			\item 2 ECTS-Punkte für Übergreifende Kompetenz
		\end{itemize}
	\end{block}
\end{frame}

\begin{frame}[fragile, t]{Organisatorisches}%{Übungen}
	\begin{block}{Übungszettel}
		\begin{itemize}
			\item Übungszettel auf \url{http://latexkurs.de/uebungen}
			\item Ausgabe Freitag nach der Vorlesung
		\end{itemize}
	\end{block}
	\begin{block}{Abgabe}
	Je nach Aufgabenstellung per E-Mail oder ausgedruckt
		\begin{description}
			\item[E-Mail] \href{mailto:abgabe@latexkurs.de}{\texttt{abgabe@latexkurs.de}}\\
			\item[Betreff] |LaTeX-Abgabe: Musterfrau, Mustermann|\\
			\item[Dateinamen] |uebung2.1_musterfrau_mustermann.tex|
		\end{description}
		\vspace{-1ex}
		\begin{itemize}
			\item Abgabe per E-Mail bis Freitag 14:00
			\item Abgabe der Ausdrucke Freitag in der Vorlesung
			\item Abgabe in Dreiergruppen möglich (ausgenommen erstes Blatt)
		\end{itemize}
	\end{block}
\end{frame}



%%% ...


\section{Inhaltsübersicht}

\begin{frame}{Inhalt (vorläufig)}
	\begin{enumerate}
		\item Einführung und grundlegende Bedienung
		\item allgemeine Formatierung und Pakete
		\item Mathematiksatz I
		\item Gleitumgebungen, Tabellen
		\item Mathematiksatz II
		\item Grafiken, Abbildungen, TikZ
		\item Diagramme
		\item umfangreiche Dokumente
		\item Bibliographien, mehrsprachiger Satz
		\item Präsentationen
		\item Briefe, Lebensläufe
		\item komplexe Makros und Befehle
	\end{enumerate}
\end{frame}

\begin{frame}{Aspekte der Vorlesung}
	\begin{block}{Nutzung von \LaTeX{}}
		\begin{itemize}
			\item \emph{Wie} erreiche ich, was ich haben will?
			\item Wie funktionieren Syntax und Semantik?
		\end{itemize}
	\end{block}
	\begin{block}{Verstehen von \LaTeX{}}
		\begin{itemize}
			\item \emph{Was} passiert, wenn ich auf den Knopf drücke?
			\item Was sind zugrundeliegende Paradigmen?
		\end{itemize}
	\end{block}
	\begin{block}{Typographie}
		\begin{itemize}
			\item \emph{Warum} macht \LaTeX{} manche Dinge so und nicht anders?
			\item Auf welche Details sollte ich achten?
		\end{itemize}
	\end{block}
\end{frame}	

\begin{frame}{Laientypografie}
	Häufig gehörte Aussage: „Typographie ist doch Geschmackssache. Ich mach das so, wie es
schön aussieht!“ \\[1em] \pause
	
	\begin{fancyquote}
		Das Selbermachen ist längst üblich, die Ergebnisse oft
		fragwürdig, weil Laien-Typografen nicht sehen, was nicht stimmt
		und nicht wissen können, worauf es ankommt. So gewöhnt man
		sich an falsche und schlechte Typografie. […] \\ Jetzt könnte der
		Einwand kommen, Typografie sei doch Geschmackssache. Wenn
		es um Dekoration ginge, könnte man das Argument vielleicht
		gelten lassen, da es aber bei Typografie in erster Linie um
		Information geht, können Fehler nicht nur stören, sondern sogar
		Schaden anrichten.
		\quoted{HPW, FF}
	\end{fancyquote}
\end{frame}


\section{\TeX}

\begin{frame}[<+->][t]{The Name of the Game}
	\begin{itemize}
		\item Programm \alert{\TeX} (Seit 1977)
			\only<1>{\\ Geschrieben von Donald E. Knuth für sein Buch\\„The Art of Computer Programming“.
			\\ „\TeX“ von griechisch τέχνη} % téchne, altgr. Fähigkeit, Kunstfertigkeit, Handwerk
		\item Makropaket \alert{plain}\TeX
			\only<2>{\\ Macht \TeX\ für normale Nutzer bedienbar.}
		\item großes Makropaket \alert{La}\TeX\ (Anfänge 1980er)
			\only<3>{\\ Von Leslie Lamport: „Lamport’s \TeX“.\\ Viele Vereinfachungen für den normalen Anwender.}
		\item aktuelle, stabile Version: \alert{La}\TeX\,\alert{2$_\varepsilon$} (1994)
			\only<4>{\\ „in einer $\varepsilon$-Umgebung von 2“…}
		\item zukünftige Entwicklung: \LaTeX{}3 \\ noch nicht eigenständig verfügbar, aber als Paket \texttt{expl3} in \LaTeXe%\\ (bietet \LaTeX{}3-Syntax für Paketautoren)
	\end{itemize}		
\end{frame}

\begin{frame}[t]{Was ist \TeX\ – und was nicht?}
	\only<1>{
		\begin{block}{Dafür ist \LaTeX{} gut geeignet …}
			\begin{itemize}
				%\item Programm, um „The Art of Computer Programming“ zu schreiben
				\item Alle Schriftstücke mit logischem Aufbau
				\begin{itemize}
					\item Naturwissenschaftliche Arbeiten (hervorragender Mathesatz)
					\item Geisteswissenschaftliche Arbeiten (hervorragende Mehrsprachigkeit, Bibliographieerstellung, 	Erstellung von Apparaten etc.)
					\item Artikel, Diplomarbeiten, Dissertationen, …
					\item Buchreihen, Briefe
					\item Präsentationen
				\end{itemize}
				\item Viel „Missbrauch“ durch kreative Paketautoren
			\end{itemize}
		\end{block}
	}\only<2>{
		\begin{block}{Dafür ist \LaTeX{} weniger gut geeignet …}
			\begin{itemize}
				\item Dokumente ohne logische Struktur
				\begin{itemize}
					\item Präsentationen (bunt, drehend, blinkend, „durcheinander“)
					\item Werbezettel
					\item Plakate 
				\end{itemize}
				\item Dokumente mit vielen uneinheitlichen Bildern, die frei bewegt werden
			\end{itemize}
		\end{block}
	}
\end{frame}

\begin{frame}{Wie funktioniert \TeX?}
	\begin{itemize}
		\item WYSIWYM
		\item reine Textdateien
		\item keine versteckten Einstellungen
		\item Textauszeichnung durch besondere Befehle:
		\begin{itemize}
			\item „Ich will einen Artikel schreiben!“
			\item „Setze eine Überschrift!“
			\item „Schreibe das folgende fett!“
			\item „Setze eine Tabelle, die …“
		\end{itemize}
	\end{itemize}
\end{frame}

\begin{frame}[t]{Wie funktioniert \TeX?}
	\begin{columns}
		\begin{column}{.45\textwidth}
			\begin{block}{Vorteile}
				\begin{itemize}
					\item Stabilität und Portabilität
					\item geringe Dateigrößen
					\item Bearbeitung mit beliebigem Editor
					\item Textdateien \emph{immer} lesbar
					\item Ausgabe überall gleich
				\end{itemize}
			\end{block}
		\end{column}
		\begin{column}{.45\textwidth}
			\begin{block}{Nachteile}
				\begin{itemize}
					\item Ergebnis nicht direkt sichtbar
					\item unintuitive Bedienung
					\item steile Lernkurve
					\item Bei Änderungen muss alles neu kompiliert werden
					\item komplizierte Layout-wünsche schwer realisierbar
				\end{itemize}
			\end{block}
		\end{column}
	\end{columns}
\end{frame}
	

\begin{frame}[fragile]{Ein einfaches \TeX-Dokument}
	Wie lässt sich Text von Befehlen unterscheiden?\\[1em]
	
	Ansatz in \emph{klassischen} Programmiersprachen:
	\begin{lstlisting}
print ( “ Hallo Welt! ” );
	\end{lstlisting}
	⇒ für ein Textsatzprogramm ungeeignet
\end{frame}
\begin{frame}[fragile]{Ein einfaches \TeX-Dokument}
	\begin{itemize}
		\item \TeX{} ist eine Auszeichnungssprache (\emph{markup language})
		\item einzelne Zeichen haben besondere Bedeutung
		\item Backslash (\verb|\|) dient als \emph{escape character} und markiert den Anfang eines Befehls: \verb|\chapter \section \author|
	\end{itemize}\quad\\[1em]
	Einfachstes \TeX-Dokument:
	\begin{lstlisting}
Hallo Welt! \bye 
	\end{lstlisting}\quad\\[1em]
	\pause
	„|tex dokument.tex|“ erzeugt ein |.dvi|-Dokument und eine |.log|-Datei
	% Live-Vorführung
\end{frame}

\begin{frame}[fragile]{Befehlszeichen}
	\begin{tabular}{ll}
			\verb|\| & \emph{escape character}, Leitet Befehle ein \\
			\verb|{}| & \emph{grouping character}, gruppieten zusammengehörende Zeichen \\& z.\,B. Argumente \verb|\textbf{fett}|\\
			\verb|$| & \emph{math character}, startet und beendet Mathemodus \\
			\verb|&| & \emph{tabbing character}, trennt Spalten in Tabellen \\
			\verb|%| & \emph{comment character} Kommentiert den Rest der Zeile aus \\
			\verb|^_~#| & weitere Zeichen mit besonderer Bedeutung
	\end{tabular}
\end{frame}

\begin{frame}[fragile]{Ein einfaches \LaTeX-Dokument}
\begin{LTXexample}
\documentclass{minimal}
\begin{document}
Hallo Welt!
\end{document}
\end{LTXexample}
\end{frame}

\begin{frame}{Dokumentenklassen}
	Dokumentenklassen legen grundlegende Eigenschaften des Dokuments fest:
	\begin{itemize}
		\item Layout
		\item Standardschriften
		\item Satzspiegel
		\item Gliederungsbefehle
		\item Aussehen von Verzeichnissen, Tabellen, Aufzählungen, …
	\end{itemize}
	Eigenschaften sind durch Änderung von Optionen oder Laden von Paketen anpassbar.
\end{frame}

\begin{frame}{Dokumentenklassen}
	\begin{tabular}{rl}
		&\kern-1.5cm \bf Standardklassen\\
		article & (Kurze) Artikel\\
		report & Reporte, Tagungsberichte\\
		book & Bücher\\
		letter & Briefe\\
		minimal & Für Minimalbeispiele\\ \\
		&\kern-1.5cm \bf KOMA-Script\\
		scrartcl & Erweiterung von article\\
		scrreprt & Erweiterung von report\\
		scrbook & Erweiterung von book\\
		scrlttr2 & Sehr mächtige Briefklasse\\ \\
		&\kern-1.5cm \bf Spezialklassen\\
		beamer & Für Präsentationen\\
		tikzposter & Wissenschaftliche Poster
	\end{tabular}
\end{frame}

\begin{frame}[fragile]{Gliederungsbefehle}
\begin{itemize}
\item Gliederungen strukturieren Dokumente,
\item ermöglichen automatische Nummerierung, Eintragung in Verzeichnisse, Kolumnentitel etc.
\item Werden von der Dokumentenklasse definiert
\item Grundstruktur im Kernel festgelegt
\item[⇒] bestimmte Elemente immer verfügbar
\end{itemize}
\vfill
\begin{lstlisting}
\part{Band I}
\chapter{Kapitel}
\section{Abschnitt}
\subsection{Unterabschnitt}
\subsubsection{Unterunterabschnitt}
\paragraph{Paragraph}
\subparagraph{Unterparagraph}
\end{lstlisting}
\end{frame}

\begin{frame}[fragile]{Pakete}
\begin{itemize}
\item Pakete bieten zusätzliche Funktionalität
\item Arbeitserleicherungen
\item Fehlerkorrekturen
\item Einbinden in der Präambel mittels
\pdfmarginpar[Help]{Die Praeambel ist alles was zwischen documentclass{} und begin{document} steht}
|\usepackage[|\meta{option(en)}|]{|\meta{paketname}|}|:
\begin{lstlisting}
\documentclass{article}
\usepackage{
	amsmath,
	hyperref,
}
\usepackage[left=2cm]{geometry}
\end{lstlisting}
\end{itemize}
\end{frame}

\iffalse
\begin{frame}{Kleine Auswahl an Paketen}
	\begin{tabular}{rl}
		graphics/x & bietet (erweiterte) Grafikunterstützung\\
		amsmath & Verbesserungen, Erweiterungen für den Mathesatz\\
		babel/polyglossia & Spachrensupport (Umbruchregeln, …)\\
		inputenc & ermöglicht verschiedene Eingabekodierungen\\
		fontenc & ermöglicht verschiedene Fontkodierungen\\
		lmodern & Stellt auf die lmodern-Schriften um\\
		fontspec & ermöglicht Nutzung von Systemschriften\\
		tikz & sehr mächtige Zeichenumgebung\\
		\vdots & \vdots
	\end{tabular}
\end{frame}
\fi

\begin{frame}[c,fragile]{Grundbefehle}{allgemein}
\overleaf{tex0001}
\hspace{-1cm}
\begin{tabular}{ll}
|\textrm{Serifen}| & \textrm{Serifen \quad Abcdxyz}\\
|\textit{kursiv}| & \textrm{\textit{kursiv \quad Abcdxyz}} \\%(\emph{nicht} |{\it kursiv}| verwenden)\\
|\textsl{geneigt}| & \textrm{\textsl{geneigt \quad Abcdxyz}}\\
|\textsf{serifenlos}| & \textsf{serifenlos \quad Abcdxyz}\\
|\textbf{fett}| & \textbf{fett \quad Abcdxyz}\\
|\texttt{Schreibmaschine}| & \texttt{Schreibmaschine \quad Abcdxyz}\\
|\textsc{Kapitälchen}| & \textsc{Kapitälchen \quad Abcdxyz}\\
|\emph{Hervorhebung}| & \emph{Hervorhebung \quad Abcdxyz}\\
|\\| & Zeilenende\\
|\par| oder Leerzeile & Absatzende\\
|$E = \frac{p^2}{2m}$| & Inline-Mathemodus: $E = \frac{p^2}{2m}$\\
|\[E = \frac{p^2}{2m}\]| & Display-Mathemodus: $\displaystyle E = \frac{p^2}{2m}$\\ % Arno: „well that’s cheated …“ ;-)
|\tableofcontents| & Produziert Inhaltsverzeichnis\\
|\today| & aktuelles Datum
\end{tabular}
\end{frame}

\begin{frame}[c,fragile]{Grundbefehle}{Schriftgrößen}
\overleaf{tex0001}
\begin{tabular}{ll}
|\tiny| & \tiny winzig \\
|\small| & \small klein \\
|\normalsize| & \normalsize normal\\
|\large| & \large groß\\
|\Large| & \Large größer\\
|\LARGE| & \LARGE noch größer\\
|\huge| & \huge riesig\\
|\Huge| & \Huge noch riesiger\\
\end{tabular}
%Manuelle Anpassung: |\fontsize{10}{12}\selectfont|
\end{frame}

\begin{frame}[c]{Hilfsdateien}
	\begin{tabular}{ll}
		& \bf Eingabe\\
		.tex & \TeX-Datei mit Dokumententext\\ \\
		& \bf Ausgabe\\
		.pdf & pdf\TeX-Ausgabe oder Umwandlung von (x)dvi\\ \\\pause
		& \bf Hilfsdateien (nur schreiben)\\
		.log & Log-Datei mit Informationen, Warnungen, Fehlern\\ \\\pause
		& \bf Hilfsdateien (schreiben und lesen)\\
		.aux & Hilfsdatei mit temporären Informationen\\
		.toc & table of contents\\
		.lof & list of figures\\
		.synctex.gz & nötig für die Sync\TeX-Funktion\\
		\vdots & \vdots
	\end{tabular}
\end{frame}

\frame{\centering \Huge Happy \TeX ing}

\end{document}