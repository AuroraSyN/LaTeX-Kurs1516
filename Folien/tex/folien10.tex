% !TEX program = xelatex
% !TEX spellcheck = de_DE

\documentclass{beamer}

\newif\ifweb
\webfalse
\webtrue

\usepackage{
	dtklogos,
	fontspec,
	graphicx,
	hologo,
	mathtools,
	microtype,
	pdfmarginpar,
	polyglossia,
	qrcode,
	shortvrb,
	showexpl,
	siunitx,
	tikz,
	pgfplots,
	xspace,
}

\usepackage{
	booktabs,
	blindtext,
}

\setmainlanguage{german}
\setotherlanguage{english}


\usepackage{comicneue,skull,CountriesOfEurope,simpsons}
\newfont{\cirth}{cirth scaled\magstep1}
\font\ransom=ransom10 scaled 1  % leider fehlen dazu die map-files
\newfont{\dancers}{dancers scaled\magstep1}

\setsansfont{Linux Biolinum O}
\setromanfont{Linux Libertine O}
\setmonofont[Scale=.95,AutoFakeSlant]{Inconsolata}
%\setmonofont[Scale=.99]{Anonymous Pro}


%% fancy-quotes %%%%%%%%%%%%%%
	\usepackage{tikz}
	\newcommand*\openquote{\makebox(25,-22){\scalebox{7}{\fontspec{Linux Biolinum O}“}}}
	\newcommand*\closequote{\makebox(25,-22){\scalebox{7}{\fontspec{Linux Biolinum O}”}}}
	\newcommand*{\OpenQuote}{\tikz[remember picture,overlay,xshift=-15pt,yshift=-10pt]
	     \node (OQ) {\openquote};}
	\newcommand*{\CloseQuote}{\tikz[remember picture,overlay,xshift=15pt,yshift=10pt]
	     \node (CQ) {\closequote};}
	\newenvironment{fancyquote}%
	{\hspace{-1em}\begin{quote}\OpenQuote\vspace*{1ex}\\\hspace*{1em}\begin{minipage}{.835\textwidth}}
	{\end{minipage}\vspace*{3.8ex}\\\hfill\CloseQuote\end{quote}}
	\newcommand\quoted[1]{\null\hfill{\scriptsize\sf #1}}
%%%%%%%%%%%%%%%%%%%%%%%%%%%%%%

\ifweb\else\renewcommand{\pdfmarginpar}[2][]{\null}\fi

%% overleaf %%%%%%%%%%%%%%%%%%
\usetikzlibrary{calc}
\tikzset{ href node/.style={alias=sourcenode,append after command={let \p1 = (sourcenode.north west),  \p2=(sourcenode.south east),\n1={\x2-\x1},\n2={\y1-\y2} in node [inner sep=0pt, outer sep=0pt,anchor=north west,at=(\p1)] {\href{#1}{\phantom{\rule{\n1}{\n2}}}}}}} % http://tex.stackexchange.com/a/36111
\newcommand{\overleaf}[1]{
	\begin{tikzpicture}[remember picture,overlay]
		\node [xshift=-1.2cm,yshift=1.5cm, href node={http://polr.me/#1}] at (current page.south east)
		{
			\scalebox{.55}{\parbox{4.1cm}{
				In Overleaf ausprobieren:\\[.8ex]
				\qrcode[height=4cm]{http://polr.me/#1}\\[1ex]
				\url{http://polr.me/#1}
			}}
		};
	\end{tikzpicture}
}
%%%%%%%%%%%%%%%%%%%%%%%%%%%%%%
\usetikzlibrary{shapes, arrows, trees}

\pgfplotsset{
	compat=1.12,
	width=7cm,
	lua backend=true,
}

\tikzset{onslide/.code args={<#1>#2}{%
  \only<#1>{\pgfkeysalso{#2}} % \pgfkeysalso doesn't change the path
}}
	
\newcommand{\meta}[1]{\textcolor{gray}{$\langle$\texttt{\textsl{#1}}$\rangle$}}
\newcommand{\pkg}[1]{\href{http://ctan.org/pkg/#1}{\alert{\texttt{#1}}}}
\newcommand{\TikZ}{Ti\textit{k}Z}
\newcommand{\BibTeX}{Bib\TeX}


\newenvironment{olcol}{
	\begin{columns}\begin{column}{.8\textwidth}
}{
	\end{column}\begin{column}{.108\textwidth}\relax\end{column}\end{columns}
}

\hypersetup{%
  unicode=true,
  pdfborder={000},
  pdftitle={Einführung in das Textsatzsystem LaTeX},
  pdfauthor={Moritz Brinkmann},
}

\lstloadlanguages{TeX}
\lstset{%
	language=TeX,
	backgroundcolor=\color[RGB]{229, 229, 239},
	basicstyle=\ttfamily\small,
	breakindent=0em,
	breaklines=true,
	commentstyle=\color{blue},
	keywordstyle=,
	identifierstyle=,
	captionpos=b,
	numbers=none,
	frame=lines,
	frameround=ffff,
	pos=r,
	rframe={single},
	explpreset={numbers=none}
}

\mode<presentation>{
	\useinnertheme{circles}
	\usecolortheme[rgb={0,0,.5}]{structure}
	\usecolortheme{whale}
	\usecolortheme{orchid}
	\beamertemplatenavigationsymbolsempty
	\setbeamercolor{alerted text}{fg=blue}
	\renewcommand{\thefootnote}{\fnsymbol{footnote}}
}


\begin{document}
\MakeShortVerb|

\title{Einführung in das Textsatzsystem \LaTeX{}}
\subtitle{Witziges und Obskures}
\author[Mo]{Moritz Brinkmann\\\url{moritz.brinkmann@iwr.uni-heidelberg.de}}
\date{8. Januar 2016}

\frame{\titlepage}


\begin{frame}{\textenglish{\TeX-Code as a source of humor}}
\TeX-Community bringt immer wieder lustigen Code und sinnlose Pakete hervor.\\Erste Anfänge: Postings von \textenglish{obfuscated Code} auf diversen \TeX-Mailinglisten und in Newsgroups\\ z.\,B. \href{http://www.ctan.org/pkg/inscrutable}{\alert{inscrutable}}, \href{http://www.ctan.org/pkg/reverxii}{\alert{reverxii}}
\end{frame}

\begin{frame}[fragile]
\begin{verbatim}
\let~\catcode~`76~`A13~`F1~`j00~`P2jdefA71F~`7113jdefPALLF
PA''FwPA;;FPAZZFLaLPA//71F71iPAHHFLPAzzFenPASSFthP;A$$FevP
A@@FfPARR717273F737271P;ADDFRgniPAWW71FPATTFvePA**FstRsamP
AGGFRruoPAqq71.72.F717271PAYY7172F727171PA??Fi*LmPA&&71jfi
Fjfi71PAVVFjbigskipRPWGAUU71727374 75,76Fjpar71727375Djifx
:76jelse&U76jfiPLAKK7172F71l7271PAXX71FVLnOSeL71SLRyadR@oL
RrhC?yLRurtKFeLPFovPgaTLtReRomL;PABB71 72,73:Fjif.73.jelse
B73:jfiXF71PU71 72,73:PWs;AMM71F71diPAJJFRdriPAQQFRsreLPAI
I71Fo71dPA!!FRgiePBt'el@ lTLqdrYmu.Q.,Ke;vz vzLqpip.Q.,tz;
;Lql.IrsZ.eap,qn.i. i.eLlMaesLdRcna,;!;h htLqm.MRasZ.ilk,%
s$;z zLqs'.ansZ.Ymi,/sx ;LYegseZRyal,@i;@ TLRlogdLrDsW,@;G
LcYlaDLbJsW,SWXJW ree @rzchLhzsW,;WERcesInW qt.'oL.Rtrul;e
doTsW,Wk;Rri@stW aHAHHFndZPpqar.tridgeLinZpe.LtYer.W,:jbye
\end{verbatim}
\end{frame}

\begin{frame}{Feuerwerk}
Datei \href{http://www.ctan.org/pkg/happy4th}{|happy4th.tex|} enthält verwirrenden Code, der als \TeX-Ausgabe ein Feuerwerk erzeugt.

\end{frame}

\begin{frame}[t,fragile]{Alternativen zu |blindtext|}
\overleaf{tex1001}
\begin{itemize}
\item \pkg{lipsum}
\item \pkg{blindtext} mit Optionen |bible| oder |pangram|
\item \pkg{kantlipsum}
\item \pkg{ptext} (Persischer Blindtext)
\item \pkg{quran} (Inhalt des Korans – auf Arabisch)
\end{itemize}
\end{frame}

\begin{frame}{Coffee Stains}
	\begin{columns}
		\begin{column}{.3\textwidth}
			Paket \alert{\href{http://hanno-rein.de/archives/349}{\texttt{coffee}}} erzeugt vektorisierte Kaffee-Flecken auf dem Dokument.
		\end{column}
		\begin{column}{.6\textwidth}
			\includegraphics[page=1,width=\textwidth]{coffee4.pdf}
		\end{column}
	\end{columns}
\end{frame}



\begin{frame}{witzige Schriftarten}
	\begin{itemize}[<+->]
		\item \pkg{comicsans}\hfill{\fontspec[Scale=.9]{Comic Sans MS} ziemlich furchtbare Schrift}
		\item \pkg{comicneue}\hfill{\comicneue neuere, nicht ganz so furchtbare Schrift}
%		\item \pkg{ransom}\hfill{\ransom kaputte Schreibmaschine}
		\item \pkg{cirth}\hfill{\cirth baliN SelN ov fuxiN, lon-d ov monia}
		\vspace{.2ex}

		\item \pkg{skull}\hfill{|\$\textbackslash skull\$|\quad\Large$\skull$}
		\item \pkg{countriesofeurope}\hfill{\Large\EUCountry{140}\EUCountry{141}\EUCountry{142}\EUCountry{143}\EUCountry{147}}
		\vspace{-.8em}

		\item \pkg{dancers}\hfill\scalebox{.6}{\dancers Sher­lock Holmes}\hspace{-2ex}\,
	\end{itemize}
\end{frame}

\begin{frame}{simpsons}
Schriftart \pkg{simpsons} bietet Simpsons-Charaktere:\\[-.5em]
|\textbackslash Homer| \uncover<2->{|\textbackslash Left\textbackslash Bart|}\hfill\Homer\uncover<2->{\Left\Bart} \\[2em]\pause\pause
Position der Pupille anpassbar:\\[-.5em]
|\textbackslash Goofy\textbackslash Lisa(1,1.6)(.85,1.6)|\hfill\Goofy\Lisa(1,1.6)(.85,1.6) \\[2em]
\pause
\hfill \Goofy\Marge(-.5,-.7)(1.5,.8) \Goofy\Maggie(1,-.6)(-.5,-.3) \quad \Left\Burns\hfill\,
\end{frame}

\begin{frame}[t]{chickenize}
	Paket \pkg{chickenize} nutzt Lua um einzelne Buchstaben im \hologo{LuaTeX}-Input zu manipulieren.
	\overleaf{tex1002}
	
	\begin{itemize}
		\item Zufälliges wechseln der Schriftart
		\item Färben einzelner Buchstaben
		\item Färben des ganzen Texts als Regenbogen
		\item Ersetzten aller Wörter durch das Wort „Chicken“
		\item Ersetzten aller Buchstaben durch zufällige Buchstaben
		\item …
	\end{itemize}
\end{frame}

\begin{frame}{Spaßige Anforderungen}
	Auf \url{tex.stackexchange.com} gibt es immer wieder witzige Fragen …
	\begin{itemize}
		\item \href{http://tex.stackexchange.com/questions/115471/using-contexts-cow-font-with-pdftex}{Using ConTeXts cow font with pdfTeX}
		\item \href{http://tex.stackexchange.com/questions/145223/how-to-draw-a-coffee-cup}{How to draw a coffee cup?}
		\item \href{http://tex.stackexchange.com/questions/63732/cute-child-friendly-document-in-latex}{Cute (child-friendly) document in LaTeX}
		\item \href{http://tex.stackexchange.com/questions/29402/how-do-i-make-my-document-look-like-it-was-written-by-a-cthulhu-worshipping-madm}{How do I make my document look like it was written by a Cthulhu-worshipping madman?}
		\item \href{http://tex.stackexchange.com/questions/74878/create-xkcd-style-diagram-in-tex}{Create xkcd style diagram in TeX}
	\end{itemize}
\end{frame}

%xkcd-style

\end{document}