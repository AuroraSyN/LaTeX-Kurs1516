% Dorkenwald, Soloninov
% Blatt 7.1;

\documentclass{scrartcl}

\usepackage[ngerman]{babel}
\usepackage[T1]{fontenc}
\usepackage[utf8]{inputenc}

\usepackage{pgfplots,tikz,xcolor}
\pgfplotsset{compat=1.3}
\pgfplotsset{grid style={dashed, magenta}}
\usetikzlibrary{patterns}

\begin{document}

\section*{Übung 7.1 Daten darstellen mit pgfplots}


\subsection*{Wie finden Sie Himbeeren?}
\begin{tikzpicture}
	\begin{axis}[ xbar,
		width=14cm, height = 5cm,
		enlarge y limits=0.2,
		enlarge x limits=0,
		xlabel={Anzahl der Antworten},
		symbolic y coords = {furchtbar, meh, ganz gut, genial, keine Ang.},
		ytick=data, grid ]
		\addplot[ draw=black, pattern=horizontal lines gray ] 
		coordinates {(9,furchtbar) (1,meh) (2,ganz gut) (186,genial) (0,keine Ang.)};
	\end{axis}
\end{tikzpicture}

\subsection*{Mögen Sie Tanzen?}

\begin{tikzpicture}
	\begin{axis}[xbar,
		width=14cm, height = 5cm,
		enlarge y limits=0.2,
		enlarge x limits=0,
		xlabel={Anzahl der Antworten},
		symbolic y coords = {furchtbar, meh, ganz gut, genial, keine Ang.},
		ytick=data, grid ]
		\addplot[ draw=black, pattern=horizontal lines gray ] 
		coordinates {(32,furchtbar) (63,meh) (52,ganz gut) (49,genial) (2,keine Ang.)};
	\end{axis}
\end{tikzpicture}

\subsection*{Was halten Sie von Topfpflanzen?}
\begin{tikzpicture}
	\begin{axis}[xbar,
		width=14cm, height = 5cm,
		enlarge y limits=0.2,
		enlarge x limits=0,
		xlabel={Anzahl der Antworten},
		symbolic y coords = {furchtbar, meh, ganz gut, genial, keine Ang.},
		ytick=data,grid ]
		\addplot[ draw=black, pattern=horizontal lines gray] 
		coordinates {(28,furchtbar) (17,meh) (12,ganz gut) (26,genial) (115,keine Ang.)};
	\end{axis}
\end{tikzpicture}

\end{document}
