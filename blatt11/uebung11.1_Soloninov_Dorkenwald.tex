%Blatt 11, Aufgabe 1
%Soloninov, Dorkenwald


\documentclass[fontsize=12pt]{scrlttr2} 
\usepackage[T1]{fontenc}
\usepackage{ngerman,ae,times, graphicx,url}
\usepackage[ngerman]{babel}
\usepackage[utf8]{inputenc}

\KOMAoptions{paper=a4,fromalign=center,fromrule=aftername,
backaddress=true,parskip=half,enlargefirstpage=true} 
 
\setkomavar{fromname}{Soloninov Aleksandr} 		
\setkomavar{fromaddress}{Kirchstrasse. 22\\
                        68623 Lampertheim-Hofheim} 

\setkomavar{signature}{Soloninov Aleksandr}		
 
\setkomavar{place}{Lampertheim-Hofheim} 	

\setkomavar{subject}{Bewerbung als ... ... ...}
 
\let\raggedsignature=\raggedright		
  
\begin{document}
 \begin{letter}{ Universität Heidelberg \\- Dezernat Internationale Beziehungen -\\
		     Seminarstraße 2\\69117 Heidelberg} 		
 
\opening{Sehr geehrte Damen und Herren,}
 
hier folgt der erste Absatz, der auch gleichzeitig die 
\textbf{Einleitung} darstellt. Am besten kommt man gleich zur 
Sache: Warum interessiert mich diese Stelle, und warum halte
ich mich für geeignet.
 
Im zweiten Absatz beginnt der \textbf{ Hauptteil}. Hier stellt 
man sich vor, und hier sollte man anhand von Qualifikationen 
und Erfahrungen belegen, warum man die Anforderungen
erfüllt. Im Hauptteil sollte man auch persönliche Qualitäten 
erwähnen: Welche Hard und Soft Skills bringe ich mit (ich bin 
teamfähig, flexibel, etc.).
 
Der letzte Absatz gehört dem \textbf{Schluss}. Hier bekundet 
man nocheinmal sein Interesse, sowie die Reaktion, die man sich 
wünscht. ("Über eine Einladung zu einem persönlichen 
Gespräch würde ich mich sehr freuen."')
 
\closing{Mit freundlichen Grüßen}
 
 
\encl{%
         Lebenslauf\\}
 
 
\end{letter}
\end{document}