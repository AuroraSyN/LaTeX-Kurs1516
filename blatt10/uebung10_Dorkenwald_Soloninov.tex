% !TEX program = pdfLaTeX
% !TEX encoding = UTF-8 Unicode
% !TEX spellcheck = de_DE


%Blatt 10
%Soloninov, Dorkenwald

\documentclass{beamer} %% normal document
%\documentclass[notes]{beamer} %% notes in normal document
%\documentclass[draft,notes]{beamer} %% draft with notes
%\documentclass[handout]{beamer} %% handout

\usepackage[ngerman]{babel}
\usepackage{showexpl} 
\usepackage[T1]{fontenc}
\usepackage[utf8]{inputenc}
\usepackage{caption}

\usepackage{enumitem}

%% usefull for handout with blank lines
%\usepackage{handoutWithNotes}
%\pgfpagesuselayout{2 on 1 with notes}[a4paper,border shrink=5mm]

%% usefull for presentation
%\setbeameroption{show notes on secondscreen=left}

\definecolor{bettergreen}{rgb}{.1,.7,.1}

\usetheme{Dresden}
\usecolortheme[named=bettergreen]{structure}
\useoutertheme{split}
\setbeamertemplate{caption}[numbered]
\captionsetup{labelformat=simple,font=scriptsize,labelfont=scriptsize}

\newcommand*\oldmacro{}%
\let\oldmacro\insertshorttitle%
\renewcommand*\insertshorttitle{%
	\oldmacro\hfill%
	\insertframenumber\,/\,\inserttotalframenumber}

\title[]{}


\author{
	Soloninov Aleksandr, Dorkenwald Michael\\
}
\institute[IFI]{
	Vorlesung: LaTex-Kurse\\
	Universität Heidelberg\\
	Präsentation ist von ISW team - b03 WS 15/16  geänder für LaTeX-Kurs 15/16
}

\begin{document}
	\begin{frame}[plain]
		\titlepage
		\note{ }
	\end{frame}

	%%\section*{Teampräsentation}
	\begin{frame}{Teampräsentation}
		\center{\huge LaTeX, Blatt 10}
		\vspace{2em}
		\begin{itemize}
			\item Soloninov Aleksandr\\
				B.Sc. Angewandte Informatik, 3. Fachsemester
			\item Dorkenwald Michael\\
				B.Sc. Physic, 5. Fachsemester\\
		\end{itemize}
	\end{frame}
	\section{Bisheriger Zustand}
	\begin{frame}{Bisheriger Zustand}
		\begin{figure}[H] 
			\centering
			\includegraphics[width=0.85\linewidth]{uebung10_Soloninov_Dorkenwald}
		\end{figure}
	\end{frame}
	
	\section{Verbesserungsvorschläge}
	\subsection{Probleme und Lösungen}
	\begin{frame}{Probleme und Lösungen}
		\begin{columns}[c]
			\begin{column}{0.5\textwidth}
				\begin{itemize}
					\item \includegraphics[height=\baselineskip]{uebung10_Soloninov_Dorkenwald} Viel manuelle Eingaben nötig
					\item \includegraphics[height=\baselineskip]{uebung10_Soloninov_Dorkenwald} Angaben möglicherweise unvollständig
					\item \includegraphics[height=\baselineskip]{uebung10_Soloninov_Dorkenwald} Angaben möglicherweise unkorrekt
				\end{itemize}
			\end{column}
			\pause
			\begin{column}{0.5\textwidth}
				\begin{itemize}
					\item \includegraphics[height=\baselineskip]{uebung10_Soloninov_Dorkenwald} Manuelle Eingabe minimiert
						\pause
					\item \includegraphics[height=\baselineskip]{uebung10_Soloninov_Dorkenwald} Eingabe autovervollständigt
						\pause
					\item \includegraphics[height=\baselineskip]{uebung10_Soloninov_Dorkenwald} unkorrekte Eingaben hervorgehoben
				\end{itemize}
			\end{column}
		\end{columns}
	\end{frame}


	\subsection{Umsetzung}
	\begin{frame}{Umsetzung}
		\small
		\begin{tabular}{p{1.7cm} p{2.6cm} p{2.6cm} p{2.3cm} }
			\textbf{Funktion} & \textbf{Beschreibung} & \textbf{Eingabe} & \textbf{Ausgabe} \pause \\
			\\
			Complete & vervollständigt mit IMDB & Titelanfang, orig. Title, Time, Category, IMDB-Url & Array of {Title, IMDB-Url} \pause \\
			\\
			Validate & prüft eingegebene Werte & Title, orig.~Title, Category, IMDB-Url, do~not~validate? & rote~Fläche, grüne~Fläche \\
		\end{tabular}
	\end{frame}

	\subsection{Verbessertes UI}
	\begin{frame}{Verbessertes UI}
		\begin{figure}[H] 
			\centering
			\includegraphics[width=0.85\linewidth]{uebung10_Soloninov_Dorkenwald}
		\end{figure}
	\end{frame}

	\section{}
	\begin{frame}
		\center{\Huge ENDE}
	\end{frame}

\end{document}
