% !TEX TS-program = lualatex
% !TEX encoding = UTF-8 Unicode
% !TEX spellcheck = de_DE

\documentclass{scrartcl}

\usepackage{polyglossia}
\setmainlanguage{german}
\usepackage{fontspec}
\setsansfont{Linux Biolinum O}
\setromanfont{Linux Libertine O}
\setmonofont[Scale=.85]{Inconsolata}

\usepackage{
	array,
	booktabs,
	caption,
	colortbl,
	enumitem,
	exsheets,
	exsheets-listings,
	geometry,
	hologo,
	hhline,
	lastpage,
	microtype,
	scrlayer-scrpage,
	showexpl,
	shortvrb,
	hyperref,
}
\newif\ifdraft

%%%%%%%%%%%%%%%%%%%%%%%%%%%%%%%%%%%%%%
\newcommand{\blattnr}{4}
\newcommand{\ausgabetermin}{06.11.2015}
\newcommand{\abgabetermin}{13.11.2015}
\draftfalse
\SetupExSheets{solution/print=true}
%%%%%%%%%%%%%%%%%%%%%%%%%%%%%%%%%%%%%%

\geometry{a4paper,vmargin=3cm,head=26.0pt}
\frenchspacing
\reversemarginpar
\hypersetup{
	colorlinks=true,
	linkcolor=blue,
	urlcolor=blue,
}
\renewcommand*{\thefootnote}{\fnsymbol{footnote}}
\pagestyle{scrheadings}
\KOMAoptions{
	headtopline = 1pt,
	headsepline = .6pt,
	footsepline = .6pt,
}
\setkomafont{pagehead}{\normalfont\small\sffamily}
\setkomafont{pagefoot}{\small\itshape}
\ihead{Einführung in das\\Textsatzsystem \LaTeX}
\chead{\bfseries Übungsblatt \blattnr}
\ohead{Ausgegeben: \ausgabetermin\\Abgabe: \abgabetermin}
\ifoot{Heidelberg, WS 2015}
\cfoot{}
\ofoot{Seite~\thepage~von~\pageref{LastPage}}

\MakeShortVerb{|}
\lstset{%
	backgroundcolor=\color[rgb]{.9 .9 .9},
	basicstyle=\ttfamily\small,
	breakindent=0em,
	breaklines=true,
	language=[LaTeX]TeX,
	commentstyle=,
	keywordstyle=,
	identifierstyle=,
	tabsize=2,
	captionpos=b,
	numbers=left,
	numberstyle=\tiny,
	stepnumber=0,
	numberfirstline=false,
	frame=single
}
\ifdraft
	\AtBeginDocument{\centerline{%
   	\color{red}\Large \sffamily Vorläufige Version}%
		\vspace*{1ex}%
	}
\fi
\newcommand{\abgabe}[1]{\par\noindent\textit{Abgabe:} #1}
\DeclareRobustCommand*\questionstar{\texorpdfstring{\bonusquestionsign}{* }}
\DeclareRobustCommand*\bonusquestionsign{\llap{$\bigstar$\space}}
\NewQuSolPair
	{question*}[name=\questionstar Bonusaufgabe]
	{solution*}[name=\questionstar Lösung]
\DeclareInstance{exsheets-heading}{myblock}{default}{
	title-pre-code = \sffamily,
	number-pre-code =  \sffamily\bfseries~\blattnr.,
	subtitle-pre-code = \upshape\sffamily\bfseries:~,
	points-pre-code = \itshape,
	join = {
		number[r,vc]subtitle[l,vc](0pt,0pt);
		title[r,vc]number[l,vc](0pt,0pt);
	} ,
	attach = { 
		main[l,vc]title[l,vc](0pt,0pt);
		main[r,vc]points[r,vc](0pt,0pt);
	},
}
\SetupExSheets{
	counter-format = qu[1],
	headings = myblock,
}
\SetupExSheets[points]{
	name = {\,Punkt/e},
	bonus-name = {Bonuspunkt/e},
}

%%%%%%%%%%%%%%%%%%%%%%%%%%%%%%%%%%%%%%

\begin{document}

\begin{question}[subtitle=Schöne Tabelle]{6}
	Mithilfe des Pakets \verb|booktabs| ist es möglich, typografisch anspruchsvolle Tabellen zu setzen. Die Dokumentation\footnote{\texttt{texdoc booktabs}} liefert viele Gestaltungshinweise und Best-Practice-Beispiele. Die folgende Tabelle hält sich leider nicht so richtig an diese Empfehlungen.

\begin{minipage}{\textwidth}
	\centering
	\begin{tabular}{|p{2.1cm}||p{2.5cm}|cl|lr|}
		\hhline{=-----}
		Produkt & Herkunft & Saisonbegin & Saisonende & Handelsklasse & verfügbar\\\hhline{======}
		Auberginen & Frankreich & Juli & September & I & – \\\hline
		Esskastanien & Frankreich & September & September & I & – \\\hline
		Feldsalat & Deutschland & Oktober & Februar & II & ja \\\hline
		Kürbis & Deutschland & August & Dezember & I & ja\\\hline
		Rote Beete & Italien & September & Februar & I  & ja\\\hline
		Zucchini & Spanien & Juni & Oktober & II &  –\\\hline
		Zwiebeln & Deutschland & Mai & Oktober & – & –\\\hhline{======}
	\end{tabular}
\end{minipage}
	
	\begin{enumerate}[label=\alph*)]
		\item Korrigieren Sie dieses Manko und setzen Sie die Tabelle nach allen Regeln der Kunst in ein \LaTeX-Dokument. Denken Sie dabei auch über die in der Vorlesung vorgestellten Möglichkeiten des \verb|multirow|-Paktets und des \texttt{\textbackslash multicolumn}-Befehls nach.
		\item Setzen Sie die Tabelle außerdem in eine geeignete Gleitumgebung und versehen Sie sie mit einem vielsagenden Titel.
		\item Geben Sie \emph{handschriftlich} einen kurzen Kommentar zu den von Ihnen vorgenommenen Änderungen an. (Grund für die Änderungen, Probleme, Alternativlösungen, …)
	\end{enumerate}
	\abgabe{Den Quelltext per Mail, Quellcode und fertiges Dokument als Ausdruck. Handschriftlicher Kommentar auf dem Ausdruck.}
\end{question}


\newpage 
\newsavebox{\SolutionCodeA}
\begin{lrbox}{\SolutionCodeA}
\begin{minipage}{\textwidth}
Eine Möglichkeit die Tabelle zu Setzen wäre zum Beispiel:

\begin{LTXexample}[pos=b]
\begin{table}
	\centering
	\begin{tabular}{llcccl}
		\toprule
		&& \multicolumn{2}{c}{\textbf{Saison}} \\
		\cmidrule{3-4}
		\textbf{Produkt} & \textbf{Herkunft} & \textbf{Begin} & \textbf{Ende} & \textbf{Handelskl.} & \textbf{Verfügbarkeit}\\
		\midrule
		Auberginen & Frankreich & Juli & September & I &  \\
		Esskastanien & Frankreich & \multicolumn{2}{c}{September} & I &  \\
		Feldsalat & Deutschland & Oktober & Februar & II & ja \\
		Kürbis & Deutschland & August & Dezember & I & ja\\
		Rote Beete & Italien & September & Februar & I  & ja\\
		Zucchini & Spanien & Juni & Oktober & II &  \\
		Zwiebeln & Deutschland & Mai & Oktober &  & \\
		\bottomrule
	\end{tabular}
	\caption{Unser saisonales Gemüseangebot}
\end{table}
\end{LTXexample}
\end{minipage}
\end{lrbox}

\begin{solution}
\noindent\usebox\SolutionCodeA
\end{solution}

\newpage 

\begin{question}[subtitle=Bunte Tabelle]{6}
	Bisher waren alle Übungsaufgaben in klassischem schwarz/weiß gehalten. An dieser Stelle soll aber gezeigt werden, dass \LaTeX\ sehr wohl auch mit Farben umgehen kann – und zwar am Beispiel einer Tabelle.

	Farben können in \LaTeX\ mit dem Paket \verb|color| verwendet werden, das den Befehl \texttt{\textbackslash color\{\}} zur Verfügung stellt. \texttt{\textbackslash color\{\}} nimmt als Argument eine bekannte Farbe und stellt auf diese um, z.\,B. \texttt{\textbackslash color\{blue\}}. Angaben in rgb-Code o.\,ä. sind auch möglich, z.\,B. \texttt{\textbackslash color[gray]\{0.8\}}. Das Paket \verb|xcolor| erweitert die Möglichkeiten des \texttt{\textbackslash color}-Befehls noch enorm. (Unter anderem ist die Angabe einer Wellenlänge statt Farbe möglich.)

	Studieren Sie die Dokumentation des Paketes \verb|colortbl| und produzieren Sie mithilfe dieses Paketes ein Minimalbeispiel, das eine Tabelle enthalten soll, die wie die folgende aussieht: % der Code ist natürlich keine Vorlage, sondern soll die geforderte Lösung mit anderer Implementierung visualisieren

	\centering
	\begin{tabular}{l!{\hspace*{-4.3mm}}l}
		\colorbox{gray}{\textcolor{black}{Wochentag}} & \colorbox{gray}{\textcolor{black}{Buchstaben\vphantom{Wg}}}\\
		\colorbox{blue}{\textcolor{white}{Montag\hspace*{.8em}}} & 6 \\
		\colorbox{blue}{\textcolor{white}{Dienstag\hspace*{.34em}}} & 8\\
		\colorbox{blue}{\textcolor{white}{Mittwoch}} &  8\\
		\vdots & \vdots \\
	\end{tabular}
	\captionof{table}{Eine wichtige Tabelle hat immer\\ eine vielsagende Beschriftung!}

	\abgabe{Den Quelltext per Mail und als Ausdruck.}
\end{question}



\newsavebox{\SolutionCodeB}
\begin{lrbox}{\SolutionCodeB}
\begin{minipage}{\textwidth}
\begin{lstlisting}
\documentclass{scrartcl}
\usepackage{colortbl}
\setlength{\tabcolsep}{.3em}

\begin{document}
  \begin{table}
    \centering
    \begin{tabular}{>{\color{white}\columncolor{blue}}l l}
      \rowcolor[gray]{.5} \color{black} Wochentag & Buchstaben\\
      Montag & 6 \\
      Dienstag & 8 \\
      Mittwoch & 8 \\
      Donnerstag & 10 \\
      Freitag & 7 \\
      Samstag & 7 \\
      Sonntag & 7 \\
    \end{tabular}
  \caption{Hier kommt die vielsagende Beschriftung}
  \end{table}
\end{document}
\end{lstlisting}
\end{minipage}
\end{lrbox}

\begin{solution}
\noindent\usebox\SolutionCodeB
\end{solution}



\end{document}