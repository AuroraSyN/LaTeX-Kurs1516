% !TEX TS-program = lualatex
% !TEX encoding = UTF-8 Unicode
% !TEX spellcheck = de_DE

\documentclass{scrartcl}

\usepackage{polyglossia}
\setmainlanguage{german}
\usepackage{fontspec}
\setsansfont{Linux Biolinum O}
\setromanfont{Linux Libertine O}
\setmonofont[Scale=.85]{Inconsolata}

\usepackage{
	enumitem,
	exsheets,
	exsheets-listings,
	geometry,
	hologo,
	lastpage,
	mathtools,
	microtype,
	scrlayer-scrpage,
	shortvrb,
	showexpl,
	hyperref,
}
\newif\ifdraft

%%%%%%%%%%%%%%%%%%%%%%%%%%%%%%%%%%%%%%
\newcommand{\blattnr}{3}
\newcommand{\ausgabetermin}{30.10.2015}
\newcommand{\abgabetermin}{06.11.2015}
\draftfalse
\SetupExSheets{solution/print=true}
%%%%%%%%%%%%%%%%%%%%%%%%%%%%%%%%%%%%%%

\geometry{a4paper,vmargin=3cm,head=26.0pt}
\frenchspacing
\reversemarginpar
\hypersetup{
	colorlinks=true,
	linkcolor=blue,
	urlcolor=blue,
}
\renewcommand*{\thefootnote}{\fnsymbol{footnote}}
\pagestyle{scrheadings}
\KOMAoptions{
	headtopline = 1pt,
	headsepline = .6pt,
	footsepline = .6pt,
}
\setkomafont{pagehead}{\normalfont\small\sffamily}
\setkomafont{pagefoot}{\small\itshape}
\ihead{Einführung in das\\Textsatzsystem \LaTeX}
\chead{\bfseries Übungsblatt \blattnr}
\ohead{Ausgegeben: \ausgabetermin\\Abgabe: \abgabetermin}
\ifoot{Heidelberg, WS 2015}
\cfoot{}
\ofoot{Seite~\thepage~von~\pageref{LastPage}}

\MakeShortVerb{|}
\lstset{%
	backgroundcolor=\color[rgb]{.9 .9 .9},
	basicstyle=\ttfamily\small,
	breakindent=0em,
	breaklines=true,
	language=[LaTeX]TeX,
	commentstyle=,
	keywordstyle=,
	identifierstyle=,
	tabsize=2,
	captionpos=b,
	numbers=left,
	numberstyle=\tiny,
	stepnumber=0,
	numberfirstline=false,
	frame=single
}
\ifdraft
	\AtBeginDocument{\centerline{%
   	\color{red}\Large \sffamily Vorläufige Version}%
		\vspace*{1ex}%
	}
\fi
\newcommand{\abgabe}[1]{\par\noindent\textit{Abgabe:} #1}
\DeclareRobustCommand*\questionstar{\texorpdfstring{\bonusquestionsign}{* }}
\DeclareRobustCommand*\bonusquestionsign{\llap{$\bigstar$\space}}
\NewQuSolPair
	{question*}[name=\questionstar Bonusaufgabe]
	{solution*}[name=\questionstar Lösung]
\DeclareInstance{exsheets-heading}{myblock}{default}{
	title-pre-code = \sffamily,
	number-pre-code =  \sffamily\bfseries~\blattnr.,
	subtitle-pre-code = \upshape\sffamily\bfseries:~,
	points-pre-code = \itshape,
	join = {
		number[r,vc]subtitle[l,vc](0pt,0pt);
		title[r,vc]number[l,vc](0pt,0pt);
	} ,
	attach = { 
		main[l,vc]title[l,vc](0pt,0pt);
		main[r,vc]points[r,vc](0pt,0pt);
	},
}
\SetupExSheets{
	counter-format = qu[1],
	headings = myblock,
}
\SetupExSheets[points]{
	name = {\,Punkt/e},
	bonus-name = {Bonuspunkt/e},
}

%%%%%%%%%%%%%%%%%%%%%%%%%%%%%%%%%%%%%%

\begin{document}

\begin{question}[subtitle=Maxwell-Gleichungen]{6}
	Jeder Physiker sollte einmal im Leben die Maxwell-Gleichungen \TeX{}en, und auch für Studis anderer Fächer bieten diese Formeln eine gute Möglichkeit, den Mathesatz zu testen. Die vier Gleichungen lauten:

	\[\textstyle \nabla \times \vec E = -\frac{\partial \vec B}{\partial t}\quad \nabla \times \vec B = \vec j + \frac{\partial \vec E}{\partial t}\quad \nabla \cdot \vec E = \rho\quad 
	\nabla \cdot \vec B = 0\]
	
	\noindent Überlegen Sie sich eine passende Formatierung und eine gute, übersichtliche Darstellung.\footnote{Die hier gezeigte Darstellung ist nicht optimal; sie sollen sich also explizit eine bessere überlegen.} Korrigieren Sie Abstände, falls nötig, wählen Sie eine gute und passende Schrift, überlegen Sie sich mögliche Auszeichnungsformen vektorieller Größen etc.
	
	Die nötigen Zeichen für diese Aufgabe finden Sie in der Datei |symbols-a4|\footnote{\texttt{texdoc symbols-a4}} oder mit dem \href{http://detexify.kirelabs.org/classify.html}{Detexify-Tool}.\footnote{\url{http://detexify.kirelabs.org/classify.html}}
	
	\abgabe{Den Quelltext per Mail und als Ausdruck.}
\end{question}

\newsavebox{\SolutionCodeA}
\begin{lrbox}{\SolutionCodeA}
\begin{minipage}{\textwidth}
Eine Möglichkeit der Auszeichnung von Vektoren wäre der Fettdruck mittels |\pmb| aus dem |amsmath|-Paket. Eine Schöne Dartsellung der vier Formeln kann man mit der |align|-Umgebung erreichen:
\begin{LTXexample}
\begin{align*}
  \nabla \times \pmb E &= -\frac{\partial \pmb B}{\partial t} &
  \nabla \times \pmb B &= \pmb j + \frac{\partial \pmb E}{\partial t}\\
  \nabla \cdot \pmb E &= \rho &
  \nabla \cdot \pmb B &= 0
\end{align*}
\end{LTXexample}
\end{minipage}
\end{lrbox}

\begin{solution}
\noindent\usebox\SolutionCodeA
\end{solution}


\begin{question}[subtitle=Integrale]{3}
	Ein befreundeter Mathematiker hat eine eine großartige neue Theorie aufgestellt, die mit folgendem Integral-Konstrukt notiert wird:
	\begin{center}
		\includegraphics{03komischeintegrale} % Das habt Ihr Euch wohl so gedacht ;-)
	\end{center}
	Helfen Sie Ihrem Mathematiker-Freund beim Setzen seiner Theorie und bauen Sie das seltsame Integral aus vorhandenen Integralsymbolen in \hologo{pdfLaTeX} nach.

	\noindent\emph{Tipp:} Suchen Sie in |symbols-a4| nach |txfonts/pxfonts| und benutzen Sie den \texttt{\textbackslash kern}-Befehl um horizontale Abstände zu verändern.
	\abgabe{Den Quelltext per Mail, das fertige PDF als Ausdruck.}
\end{question}

\newsavebox{\SolutionCodeB}
\begin{lrbox}{\SolutionCodeB}
\begin{minipage}{\textwidth}
Das Integral-Konstrukt kann mit folgendem Code erzeugt werden:
\begin{lstlisting}
\documentclass{minimal}
  \usepackage{amsmath,pxfonts}
\begin{document}
  \begin{displaymath}
    \oint\kern.15em\vec\varoiintctrclockwise\kern-2.60em\varoiintclockwise
  \end{displaymath}
\end{document}
\end{lstlisting}
\end{minipage}
\end{lrbox}

\begin{solution}
\noindent\usebox\SolutionCodeB
\end{solution}


\newsavebox{\QuestCodeC}
\begin{lrbox}{\QuestCodeC}
\begin{minipage}{\textwidth}
\begin{lstlisting}
\[ a =
  \begin{cases}
	  b & b > 0 \\
    -b & b < 0\\
    0 & b = 0
  \end{cases} 
\]
\end{lstlisting} 
\end{minipage}
\end{lrbox}

\begin{question}[subtitle=Fallunterscheidungen mit Cases]{3}
	Die Umgebung |cases| ermöglicht den Satz von Fallunterscheidungen:
	\\\noindent\usebox\QuestCodeC\\
	Das \emph{muss} zwar im Mathesatz passieren, kann aber auch für Textinhalte (im Mathemodus mit \texttt{\textbackslash text\{normaler Text\}} zu setzen) nützlich sein. Alternativ kann man auch eine |matrix|-Umgebung verwenden und die nötige Klammer von Hand (z.\,B. \texttt{\textbackslash left(}) setzen. Verwenden Sie nun letzteres, um zwei Formeln aus dem tractatus philosophicus in folgender Form zu setzen:
	\begin{displaymath}
		\text{tractatus philosophicus, Satz 6.}
		\begin{cases}
			03:\hspace*{0.65em} [o,\xi, \xi +1]\\
			231:\ 1+1+1+1 = (1+1) + (1+1)
		\end{cases}
	\end{displaymath}
	\abgabe{Den Quelltext per Mail, das fertige PDF als Ausdruck.}
\end{question}

\newsavebox{\SolutionCodeC}
\begin{lrbox}{\SolutionCodeC}
\begin{minipage}{\textwidth}
Man muss |\left\{| und |\right.| verwenden um die |cases|-Umgebung zu emulieren. Die Umgebung |matrix*| aus dem |mathtools|-Paket erlaubt es, durch Angabe des optionalen Parameters |[l]|, den Inhalt der Matrix linksbündig zu setzen.
\begin{lstlisting}
\begin{displaymath}
  \text{tractatus philosophicus, Satz 6.}
  \left\{
  \begin{matrix*}[l]
    03:\hspace*{0.65em} [o,\xi, \xi +1]\\
    231:\ 1+1+1+1 = (1+1) + (1+1)
  \end{matrix*}
  \right.
\end{displaymath}
\end{lstlisting}
\end{minipage}
\end{lrbox}

\begin{solution}
\noindent\usebox\SolutionCodeC
\end{solution}



\end{document}