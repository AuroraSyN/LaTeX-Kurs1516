% !TEX TS-program = lualatex
% !TEX encoding = UTF-8 Unicode
% !TEX spellcheck = de_DE

\documentclass{scrartcl}

\usepackage{polyglossia}
\setmainlanguage{german}
\usepackage{fontspec}
\setsansfont{Linux Biolinum O}
\setromanfont{Linux Libertine O}
\setmonofont[Scale=.85]{Inconsolata}

\usepackage{
	booktabs,
	caption,
	enumitem,
	exsheets,
	exsheets-listings,
	geometry,
	hologo,
	lastpage,
	mathtools,
	microtype,
	scrlayer-scrpage,
	shortvrb,
	siunitx,
	xspace,
	hyperref,
}
\newif\ifdraft

%%%%%%%%%%%%%%%%%%%%%%%%%%%%%%%%%%%%%%
\newcommand{\blattnr}{11}
\newcommand{\ausgabetermin}{22.01.2016}
\newcommand{\abgabetermin}{29.01.2016}
\draftfalse
\SetupExSheets{solution/print=false}
%%%%%%%%%%%%%%%%%%%%%%%%%%%%%%%%%%%%%%

\geometry{a4paper,vmargin=3cm,head=26.0pt}
\frenchspacing
\reversemarginpar
\hypersetup{
	colorlinks=true,
	linkcolor=blue,
	urlcolor=blue,
}
\renewcommand*{\thefootnote}{\fnsymbol{footnote}}
\pagestyle{scrheadings}
\KOMAoptions{
	headtopline = 1pt,
	headsepline = .6pt,
	footsepline = .6pt,
}
\setkomafont{pagehead}{\normalfont\small\sffamily}
\setkomafont{pagefoot}{\small\itshape}
\ihead{Einführung in das\\Textsatzsystem \LaTeX}
\chead{\bfseries Übungsblatt \blattnr}
\ohead{Ausgegeben: \ausgabetermin\\Abgabe: \abgabetermin}
\ifoot{Heidelberg, WS 2015}
\cfoot{}
\ofoot{Seite~\thepage~von~\pageref{LastPage}}

\MakeShortVerb{|}
\lstset{%
	backgroundcolor=\color[rgb]{.9 .9 .9},
	basicstyle=\ttfamily\small,
	breakindent=0em,
	breaklines=true,
	language=[LaTeX]TeX,
	commentstyle=,
	keywordstyle=,
	identifierstyle=,
	tabsize=2,
	captionpos=b,
	numbers=left,
	numberstyle=\tiny,
	stepnumber=0,
	numberfirstline=false,
	frame=single
}
\ifdraft
	\AtBeginDocument{\centerline{%
   	\color{red}\Large \sffamily Vorläufige Version}%
		\vspace*{1ex}%
	}
\fi
\newcommand{\abgabe}[1]{\par\noindent\textit{Abgabe:} #1}
\DeclareRobustCommand*\questionstar{\texorpdfstring{\bonusquestionsign}{* }}
\DeclareRobustCommand*\bonusquestionsign{\llap{$\bigstar$\space}}
\NewQuSolPair
	{question*}[name=\questionstar Bonusaufgabe]
	{solution*}[name=\questionstar Lösung]
\DeclareInstance{exsheets-heading}{myblock}{default}{
	title-pre-code = \sffamily,
	number-pre-code =  \sffamily\bfseries~\blattnr.,
	subtitle-pre-code = \upshape\sffamily\bfseries:~,
	points-pre-code = \itshape,
	join = {
		number[r,vc]subtitle[l,vc](0pt,0pt);
		title[r,vc]number[l,vc](0pt,0pt);
	} ,
	attach = { 
		main[l,vc]title[l,vc](0pt,0pt);
		main[r,vc]points[r,vc](0pt,0pt);
	},
}
\SetupExSheets{
	counter-format = qu[1],
	headings = myblock,
}
\SetupExSheets[points]{
	name = {\,Punkt/e},
	bonus-name = {Bonuspunkt/e},
}

\newcommand{\meta}[1]{\textcolor{gray}{$\langle$\texttt{\textsl{#1}}$\rangle$}}
\newcommand{\pkg}[1]{\href{http://ctan.org/pkg/#1}{\texttt{#1}}}
\newcommand{\TikZ}{Ti\textit{k}Z\xspace}


%%%%%%%%%%%%%%%%%%%%%%%%%%%%%%%%%%%%%%

\begin{document}
\begin{abstract}
\noindent Dies ist das letzte Übungsblatt und damit die letzte Chance Punkte für den Scheinerwerb zu erhalten. Eine kurze Übersicht zu den Notengrenzen finden Sie auf der \href{http://latexkurs.de/}{Vorlesungshomepage}. Sie haben den Kurs demnach bestanden wenn Sie insgesamt 66 oder mehr Punkte erreicht haben.

Falls Sie einzelne Übungszettel bisher noch nicht abgegeben haben und falls die Punkte dieser Zetteln relevant für Sie sein sollten (z.\,B. um die Bestehensgrenze zu erreichen), reichen Sie diese bitte bist \emph{spätestens Freitag den 29.01.} in der Vorlesung nach und geben Sie eine gute Begründungen, warum Ihnen eine fristgerechte Abgabe nicht möglich war.
\end{abstract}

\noindent Ein bekannter Verlag sucht für hochwertige Veröffentlichungen einen Kenner des Satzsystemes \LaTeX. Da Sie gerade auf der Suche nach einem neuen Arbeitsplatz sind, sollten Sie diese Chance ergreifen und sich für die Stelle melden.\vspace{1.6ex}


\begin{question}[subtitle=Anschreiben]{6}
	Kontaktieren Sie zunächst den Verlag mit einem Motivationsschreiben, also einem Brief an den Empfänger, der kurz begründet, warum Sie hervorragend geeignet sind.

Da es um die Suche nach einem \LaTeX-Spezialisten geht, sollte der Brief formal für sich sprechen. Auf den Inhalt müssen Sie also keinerlei Wert legen, aber das Aussehen sollte Ihren erfahrenen Umgang mit \LaTeX\ zeigen.

Erstellen Sie also einen Brief, der Absender, Empfänger, Datum, Titel, Anrede etc. enthält. Außerdem soll eine Anlage (siehe Aufgabe \ref{lebenslauf}) angezeigt werden. Verwenden Sie die Klasse |scrlttr2| aus dem KOMA-Bundle. Sollte Ihnen eine andere Briefklasse lieber sein, können Sie diese verwenden, wenn Sie deren Vorteile gegenüber |scrlttr2| ausführen.
\abgabe{Quelltext per Mail, Quelltext und fertiges Dokument ausgedruckt.}
\end{question}

\begin{question}[subtitle=Lebenslauf]{6}\label{lebenslauf}
	Neben dem Motivationsschreiben gehört zu einer Bewerbung selbstverständlich ein Lebenslauf. Fertigen Sie einen solchen an unter Verwendung einer der in der Vorlesung besprochenen Klassen. Hier gilt ebenfalls: Das Dokument spricht für sich selbst. Sie müssen also dem Inhalt keine Beachtung schenken und können beliebig (un-)sinnige Sachen schreiben.
\abgabe{Qelltext per Mail, Quelltext und fertiges Dokument ausgedruckt.}
\end{question}


\end{document}