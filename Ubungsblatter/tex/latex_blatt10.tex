% !TEX TS-program = lualatex
% !TEX encoding = UTF-8 Unicode
% !TEX spellcheck = de_DE

\documentclass{scrartcl}

\usepackage{polyglossia}
\setmainlanguage{german}
\usepackage{fontspec}
\setsansfont{Linux Biolinum O}
\setromanfont{Linux Libertine O}
\setmonofont[Scale=.85]{Inconsolata}

\usepackage{
	booktabs,
	caption,
	enumitem,
	exsheets,
	exsheets-listings,
	geometry,
	hologo,
	lastpage,
	mathtools,
	microtype,
	scrlayer-scrpage,
	shortvrb,
	siunitx,
	xspace,
	hyperref,
}
\newif\ifdraft

%%%%%%%%%%%%%%%%%%%%%%%%%%%%%%%%%%%%%%
\newcommand{\blattnr}{10}
\newcommand{\ausgabetermin}{15.01.2016}
\newcommand{\abgabetermin}{22.01.2016}
\draftfalse
\SetupExSheets{solution/print=false}
%%%%%%%%%%%%%%%%%%%%%%%%%%%%%%%%%%%%%%

\geometry{a4paper,vmargin=3cm,head=26.0pt}
\frenchspacing
\reversemarginpar
\hypersetup{
	colorlinks=true,
	linkcolor=blue,
	urlcolor=blue,
}
\renewcommand*{\thefootnote}{\fnsymbol{footnote}}
\pagestyle{scrheadings}
\KOMAoptions{
	headtopline = 1pt,
	headsepline = .6pt,
	footsepline = .6pt,
}
\setkomafont{pagehead}{\normalfont\small\sffamily}
\setkomafont{pagefoot}{\small\itshape}
\ihead{Einführung in das\\Textsatzsystem \LaTeX}
\chead{\bfseries Übungsblatt \blattnr}
\ohead{Ausgegeben: \ausgabetermin\\Abgabe: \abgabetermin}
\ifoot{Heidelberg, WS 2015}
\cfoot{}
\ofoot{Seite~\thepage~von~\pageref{LastPage}}

\MakeShortVerb{|}
\lstset{%
	backgroundcolor=\color[rgb]{.9 .9 .9},
	basicstyle=\ttfamily\small,
	breakindent=0em,
	breaklines=true,
	language=[LaTeX]TeX,
	commentstyle=,
	keywordstyle=,
	identifierstyle=,
	tabsize=2,
	captionpos=b,
	numbers=left,
	numberstyle=\tiny,
	stepnumber=0,
	numberfirstline=false,
	frame=single
}
\ifdraft
	\AtBeginDocument{\centerline{%
   	\color{red}\Large \sffamily Vorläufige Version}%
		\vspace*{1ex}%
	}
\fi
\newcommand{\abgabe}[1]{\par\noindent\textit{Abgabe:} #1}
\DeclareRobustCommand*\questionstar{\texorpdfstring{\bonusquestionsign}{* }}
\DeclareRobustCommand*\bonusquestionsign{\llap{$\bigstar$\space}}
\NewQuSolPair
	{question*}[name=\questionstar Bonusaufgabe]
	{solution*}[name=\questionstar Lösung]
\DeclareInstance{exsheets-heading}{myblock}{default}{
	title-pre-code = \sffamily,
	number-pre-code =  \sffamily\bfseries~\blattnr.,
	subtitle-pre-code = \upshape\sffamily\bfseries:~,
	points-pre-code = \itshape,
	join = {
		number[r,vc]subtitle[l,vc](0pt,0pt);
		title[r,vc]number[l,vc](0pt,0pt);
	} ,
	attach = { 
		main[l,vc]title[l,vc](0pt,0pt);
		main[r,vc]points[r,vc](0pt,0pt);
	},
}
\SetupExSheets{
	counter-format = qu[1],
	headings = myblock,
}
\SetupExSheets[points]{
	name = {\,Punkt/e},
	bonus-name = {Bonuspunkt/e},
}

\newcommand{\meta}[1]{\textcolor{gray}{$\langle$\texttt{\textsl{#1}}$\rangle$}}
\newcommand{\pkg}[1]{\href{http://ctan.org/pkg/#1}{\texttt{#1}}}
\newcommand{\TikZ}{Ti\textit{k}Z\xspace}


%%%%%%%%%%%%%%%%%%%%%%%%%%%%%%%%%%%%%%

\begin{document}


\begin{question}[subtitle=Präsentation mit beamer]{12+1}
	Erstellen Sie mit der Dokumentenklasse \pkg{beamer} eine Bildschirmpräsentation, die ein paar Beispiele zum Umgang mit \LaTeX\ zeigt. Gehen Sie dabei folgendermaßen vor:
\begin{enumerate}[label=\alph*)]
	\item Erstellen Sie zunächst eine Präsentation über die Arbeit mit \LaTeX, die mindestens drei Folien enthält. Jede Folie \emph{soll} einen Titel und \emph{kann} einen Untertitel haben. Der genaue Inhalt der Folien ist dabei unerheblich.
	\item Verwenden Sie auf der \emph{dritten} Folie die |verbatim|-Umgebung. In dieser soll ein kurzer Beispielcode für einen \LaTeX-Befehl gezeigt werden (etwas wie |\textbackslash textsf|, |\textbackslash color|, |\textbackslash includegraphics| o.\,ä.).
	
	Schreiben Sie mindestens drei Codezeilen in diesem Beispiel. Falls noch Zeit bleibt, geben Sie auf der gleichen Folie ein paar erklärende Stichworte zum Code.

Bedenken Sie Eigenheiten der |beamer|-Klasse beim Umgang mit |verbatim|!
	\item Ersetzen Sie nun den Inhalt der \emph{ersten beiden} Folien durch eine Aufzählung bzw. eine Nummerierung. (|itemize|, |enumerate|) Arbeiten Sie dann mit dynamischen Einblendeffekten Ihrer Wahl, d.\,h. lassen Sie Punkte nacheinander erscheinen, bleiben, verschwinden etc. Die Punkte sollten eine Vorbereitung auf die dritte Folie beinhalten, also z.\,B. kurze Erläuterungen, was \LaTeX\ ist und wie man es verwendet, was der Befehl auf der dritten Folie bezwecken soll etc. Verwenden sie jeweils mindestens 3 Punkte auf beiden Folien.
	\item Bei drei Folien ist eine Strukturierung des Dokumentes überflüssig, dennoch kann diese hier geübt werden: Fügen Sie vor jeder Folie eine |\textbackslash section| mit (sinnvollem) Namen ein und erstellen Sie eine zusätzliche Folie (vor der ersten) mit einem Inhaltsverzeichnis. Erstellen Sie weiterhin eine Folie ganz am Anfang, die eine Titelseite zeigt. Bedenken Sie, dass dazu der Autor und der Titel der Präsentation festgelegt werden müssen.
	\item Ändern Sie das Aussehen Ihrer Präsentation unter Verwendung beliebiger |theme|s, |inner| und |outer| |theme|s sowie |colortheme|s. Konsultieren Sie hierzu die |beamer|-Dokumentation, die einige Beispiele zeigt. Bedenken Sie dabei, was für alle Präsentationen gilt: Weniger ist mehr! 

Geben Sie handschriftlich einen kurzen Kommentar, warum Sie sich für welches Aussehen entschieden haben.
\end{enumerate}
\emph{Bonusaufgabe:}
Verwendet man für \LaTeX-Codebeispiele einfach die |verbatim|-Umgebung, ist die Ausgabe des Codes nicht sichtbar. Gerade für \LaTeX-Code ist es aber sehr hilfreich, Eingabe und fertige Ausgabe nebeneinander vergleichen zu können. Genau für diesen Zweck ist das Paket \pkg{showexpl} geeignet.
\begin{enumerate}[label=\alph*), resume]
	\item Verwenden Sie in Ihrer Präsentation also anstelle der |verbatim|-Umgebung die entsprechende Version aus dem \pkg{showexpl}-Paket. Konsultieren Sie dazu dessen Dokumentation.
\end{enumerate}
	\abgabe{Quellcode ausgedruckt, Quellcode und fertiges Dokument per Mail.}
\end{question}




\end{document}