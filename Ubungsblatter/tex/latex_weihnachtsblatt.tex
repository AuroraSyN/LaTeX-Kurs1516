% !TEX TS-program = lualatex
% !TEX encoding = UTF-8 Unicode
% !TEX spellcheck = de_DE

\documentclass{scrartcl}

\usepackage{polyglossia}
\setmainlanguage{german}
\usepackage{fontspec}
\setsansfont{Linux Biolinum O}
\setromanfont{Linux Libertine O}
\setmonofont[Scale=.85]{Inconsolata}

\usepackage{
	amssymb,
	booktabs,
	caption,
	enumitem,
	exsheets,
	exsheets-listings,
	geometry,
	hologo,
	lastpage,
	mathtools,
	microtype,
	scrlayer-scrpage,
	shortvrb,
	siunitx,
	xspace,
	hyperref,
}
\newif\ifdraft

%%%%%%%%%%%%%%%%%%%%%%%%%%%%%%%%%%%%%%
\newcommand{\blattnr}{W}
\newcommand{\ausgabetermin}{18.12.2015}
\newcommand{\abgabetermin}{15.01.2016}
\draftfalse
\SetupExSheets{solution/print=true}
%%%%%%%%%%%%%%%%%%%%%%%%%%%%%%%%%%%%%%

\geometry{a4paper,vmargin=3cm,head=26.0pt}
\frenchspacing
\reversemarginpar
\hypersetup{
	colorlinks=true,
	linkcolor=blue,
	urlcolor=blue,
}
\renewcommand*{\thefootnote}{\fnsymbol{footnote}}
\pagestyle{scrheadings}
\KOMAoptions{
	headtopline = 1pt,
	headsepline = .6pt,
	footsepline = .6pt,
}
\setkomafont{pagehead}{\normalfont\small\sffamily}
\setkomafont{pagefoot}{\small\itshape}
\ihead{Einführung in das\\Textsatzsystem \LaTeX}
\chead{\bfseries Weihnachtsblatt}%Übungsblatt \blattnr}
\ohead{Ausgegeben: \ausgabetermin\\Abgabe: \abgabetermin}
\ifoot{Heidelberg, WS 2015}
\cfoot{}
\ofoot{Seite~\thepage~von~\pageref{LastPage}}

\MakeShortVerb{|}
\lstset{%
	backgroundcolor=\color[rgb]{.9 .9 .9},
	basicstyle=\ttfamily\small,
	breakindent=0em,
	breaklines=true,
	language=[LaTeX]TeX,
	commentstyle=,
	keywordstyle=,
	identifierstyle=,
	tabsize=2,
	captionpos=b,
	numbers=left,
	numberstyle=\tiny,
	stepnumber=0,
	numberfirstline=false,
	frame=single
}
\ifdraft
	\AtBeginDocument{\centerline{%
   	\color{red}\Large \sffamily Vorläufige Version}%
		\vspace*{1ex}%
	}
\fi
\newcommand{\abgabe}[1]{\par\noindent\textit{Abgabe:} #1}
\DeclareRobustCommand*\questionstar{\texorpdfstring{\bonusquestionsign}{* }}
\DeclareRobustCommand*\bonusquestionsign{\llap{$\bigstar$\space}}
\NewQuSolPair
	{bonusquestion}[name=\questionstar Bonusaufgabe]
	{bonussolution}[name=\questionstar Lösung]
\DeclareInstance{exsheets-heading}{myblock}{default}{
	title-pre-code = \sffamily,
	number-pre-code =  \sffamily\bfseries~\blattnr.,
	subtitle-pre-code = \upshape\sffamily\bfseries:~,
	points-pre-code = \itshape,
	join = {
		number[r,vc]subtitle[l,vc](0pt,0pt);
		title[r,vc]number[l,vc](0pt,0pt);
	} ,
	attach = { 
		main[l,vc]title[l,vc](0pt,0pt);
		main[r,vc]points[r,vc](0pt,0pt);
	},
}
\SetupExSheets{
	counter-format = qu[1],
	headings = myblock,
}
\SetupExSheets[points]{
	name = {\,Punkt/e},
	bonus-name = {Bonuspunkt/e},
}

\newcommand{\meta}[1]{\textcolor{gray}{$\langle$\texttt{\textsl{#1}}$\rangle$}}
\newcommand{\pkg}[1]{\href{http://ctan.org/pkg/#1}{\texttt{#1}}}
\newcommand{\TikZ}{Ti\textit{k}Z\xspace}


%%%%%%%%%%%%%%%%%%%%%%%%%%%%%%%%%%%%%%

\begin{document}

\begin{abstract}
	\noindent Die folgenden Aufgaben haben alle mehr oder weniger viel mit der Vorlesung zu tun … Sie sollen Ihnen aber die Gelegenheit bieten, sich – falls Ihnen über die Weihnachtstage langweilig werden sollte – ein wenig mehr mit Ihrem neuen Lieblings\/text\/satz\/system, zu beschäftigen. 
	
	Alle Aufgaben sind Bonusaufgaben – dafür sind sie teilweise recht anspruchsvoll. Punkte werden vor allem für besonders kreative Lösungen und Ansätze vergeben. Sollten Sie über Weihnachten lieber Zeit mit Ihrer Familie oder Ihrem Kater verbringen, verpassen Sie auch nichts, wenn Sie die Aufgaben nicht bearbeiten.
\end{abstract}


\begin{bonusquestion}[subtitle=\TeX nische Weihnachtsdekoration]{+6}
	\begin{enumerate}[label=\alph*)]
		\item Zeigen Sie Ihre Zeichenkünste und erstellen Sie ein – möglichst schönes und \TeX nisch anspruchsvolles – weihnachtliches Motiv mit \TikZ. Ihrer Kreativität sind keine Grenzen gesetzt.
		\item Wenn Sie glauben Ihr Bild könnte es mit denen auf \href{http://www.texample.net/tikz/examples/nontech/christmas/}{\TeX ample.net} aufnehmen, schicken Sie den Sourcecode ruhig an den Betreiber, Stefan Kottwitz, mit dem Vorschlag, auch Ihr Beispiel aufzunehmen.
		\item Alternativ können Sie versuchen, Ihrem Bild Leben einzuhauchen: Mit dem Paket \pkg{animate} lassen sich sich einfache Animationen aus einzelnen Frames erstellen\footnote{Leider werden diese Animationen nicht in jedem PDF-Viewer korrekt angezeigt.}.
		
		 Falls sich Ihr Motiv dazu eignet, erstellen Sie mehrere Versionen, die sich leicht unterscheiden und fügen diese später zu einer Animation zusammen.\footnote{Ein PDF mit den genau den Dimensionen des Inhalts (also ohne Papierformat und ohne Rand) können Sie mit der Dokumentenklasse |standalone| erstellen.}
	\end{enumerate}
	\abgabe{Quellcode und fertiges Dokument ausgedruckt, Quellcode per Mail.}
\end{bonusquestion}

\begin{bonusquestion}[subtitle=Quine]{+6}
	Eine besondere Freude für die Programmiererin und den Programmierer sind Programme, die ihren eigenen Code ausgeben, ohne dabei eine Eingabe zu benötigen. – sogenannte Quines. Probieren Sie mit \TeX\ einen Quine zu erzeugen! Die \TeX-Datei soll also im pdf genau den Code ausgeben, mit dem sie selbst geschrieben ist. Es sollen keine zusätzlichen Befehle im Quellcode stehen, die nicht im pdf als Ausgabe erscheinen!
	\abgabe{Quellcode und fertiges Dokument ausgedruckt, Quellcode per Mail.}
\end{bonusquestion}

\newsavebox{\SolutionCodeA}
\begin{lrbox}{\SolutionCodeA}
\begin{minipage}{\textwidth}
\begin{lstlisting}
\tt\obeylines
\nopagenumbers\let~\string
\def\a{~\tt~\obeylines
~\nopagenumbers~\let~~~\string
~\def~\a~{\b~}
~\def~\b~{~{~\def~~~{~\string~~~\string~}~\def~\b~{~\string~\b~}~\a~}~}
~\a~\b~ye}
\def\b{{\def~{\string~\string}\def\b{\string\b}\a}}
\a\bye
\end{lstlisting}
\end{minipage}
\end{lrbox}

\newsavebox{\SolutionCodeB}
\begin{lrbox}{\SolutionCodeB}
\begin{minipage}{\textwidth}
\begin{lstlisting}
\documentclass{article}
\usepackage[T1]{fontenc}
\begin{document}
\obeylines\thispagestyle{empty}
\let~\textbackslash
\newcommand{\asciitilde}{\raisebox{-.5\height}{\textasciitilde}}
\newcommand{\mymyself}{{\let~\asciitilde
\renewcommand{\{}{\textbackslash\textbraceleft}
\renewcommand{\}}{\textbackslash\textbraceright}
\renewcommand{\asciitilde}{\textbackslash asciitilde}
\renewcommand{\mymyself}{\textbackslash mymyself}\myself}}
\newcommand{\myself}{
~documentclass\{article\}
~usepackage[T1]\{fontenc\}
~begin\{document\}
~obeylines~thispagestyle\{empty\}
~let\asciitilde~textbackslash
~newcommand\{~asciitilde\}\{~raisebox\{-.5~height\}\{~textasciitilde\}\}
~newcommand\{~mymyself\}\{\{~let\asciitilde~asciitilde
~renewcommand\{~\{\}\{~textbackslash~textbraceleft\}
~renewcommand\{~\}\}\{~textbackslash~textbraceright\}
~renewcommand\{~asciitilde\}\{~textbackslash asciitilde\}
~renewcommand\{~mymyself\}\{~textbackslash mymyself\}~myself\}\}
~newcommand\{~myself\}\{\mymyself\}
~myself
~end\{document\}}
\myself
\end{document}
\end{lstlisting}
\end{minipage}
\end{lrbox}

\begin{solution}
\noindent Auch wenn \TeX\ theoretisch auf den eigenen Quellcode zugreifen könnte um diesen einfach als verbatim-input einzulesen, wäre das genau genommen kein echter Quine.  Um einen Quine zu erstellen müssen wir Makros definieren, die den gewünschten Code enthalten und die wir sowohl zum Ausführen des Codes als auch für die Ausgabe desselben aufrufen können.\footnote{Eine relativ detaillierte Beschreibung von Quines (und etwas theoretischen Hintergrund) findet man auf der Seite. \url{http://www.madore.org/~david/computers/quine.html}}

In plain\TeX\ könnte ein Quine beispielsweise so aussehen:\footnote{Quelle: \url{http://tex.stackexchange.com/a/93930}}\\
\noindent\usebox\SolutionCodeA

\noindent Eine in \LaTeX implementierte Alternative wäre zum Beispiel:\footnotemark[\value{footnote}]\\
\noindent\usebox\SolutionCodeB
\end{solution}


\begin{bonusquestion}[subtitle=\TeX-Adventskalender]{+6}
	Erstellen Sie einen Adventskalender – also Dokument, das, je nach dem an welchem Datum es kompiliert wird, etwas anderes beinhaltet. Um zu unterscheiden welcher Tag ist können Sie zum Beispiel \hologo{LuaLaTeX} verwenden und mit |\textbackslash directlua\{|\meta{lua-Code}|\}| beliebigen lua-Code ausführen, oder das Paket \pkg{ifthen} zu Hilfe nehmen, mit dem man if-Statements in \LaTeX\ erzeugen kann.
	\abgabe{Quellcode und fertiges Dokument ausgedruckt, Quellcode per Mail.}
\end{bonusquestion}

\newsavebox{\SolutionCodeC}
\begin{lrbox}{\SolutionCodeC}
\begin{minipage}{\textwidth}
\begin{lstlisting}
\ifthenelse{\month = 12}{Es ist Dezember!}{Im Moment ist \emph{nicht} Dezember.}
\end{lstlisting}
\end{minipage}
\end{lrbox}

\begin{solution}
\noindent Mit den \TeX-Primitiven \texttt{\textbackslash day} und \texttt{\textbackslash month} lässt sich direkt auf Tag und Monat des aktuellen Datums zugreifen. Das Paket \pkg{ifthen} bietet praktische Makros für die Arbeit mit if-Conditions. Damit lässt sich zum Beispiel abfragen, ob gerade Dezember ist:\\
\noindent\usebox\SolutionCodeC

\noindent Das erste Argument von \texttt{\textbackslash ifthenelse} stellt dabei eine Bedingung dar, die entweder wahr oder falsch sein kann. Ist die Bedingung wahr, so wird der Inhalt des ersten, ist sie falsch, der Inhalt des zweiten Arguments ausgegeben. Aus solchen \texttt{\textbackslash ifthenelse}-Konstruktionen lässt sich jetzt leicht z.\,B. ein Adventskalender erstellen.
\end{solution}



\begin{bonusquestion}[subtitle=Weihnachts-Rätselheft]{+6}
	Ein beliebter Ferienspaß ist das Lösen von Kreuzworträtseln, Sudokus und sonstigen Spielchen. Überraschen Sie Ihre Lieben doch mal mit einem selbst ge\TeX ten Rätselheft. 
	
	Verschaffen Sie sich einen Überblick, welche Pakete es auf \href{http://ctan.org/topic/games}{CTAN} gibt, die für den Satz solcher Rätsel geeignet sind und erstellen Sie damit ein kleines Rätselheft, das einem eine Weile die Zeit vertreiben kann.
	\abgabe{Quellcode und fertiges Dokument ausgedruckt, Quellcode per Mail.}
\end{bonusquestion}






\end{document}