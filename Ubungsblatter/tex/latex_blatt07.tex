% !TEX TS-program = lualatex
% !TEX encoding = UTF-8 Unicode
% !TEX spellcheck = de_DE

\documentclass{scrartcl}

\usepackage{polyglossia}
\setmainlanguage{german}
\usepackage{fontspec}
\setsansfont{Linux Biolinum O}
\setromanfont{Linux Libertine O}
\setmonofont[Scale=.85]{Inconsolata}

\usepackage{
	booktabs,
	caption,
	enumitem,
	exsheets,
	exsheets-listings,
	geometry,
	hologo,
	lastpage,
	mathtools,
	microtype,
	pgfplots,
	scrlayer-scrpage,
	shortvrb,
	showexpl,
	siunitx,
	subcaption,
	subfloat,
	tikz,
	xspace,
	hyperref,
}
\newif\ifdraft

%%%%%%%%%%%%%%%%%%%%%%%%%%%%%%%%%%%%%%
\newcommand{\blattnr}{7}
\newcommand{\ausgabetermin}{27.11.2015}
\newcommand{\abgabetermin}{04.12.2015}
\draftfalse
\SetupExSheets{solution/print=true}
%%%%%%%%%%%%%%%%%%%%%%%%%%%%%%%%%%%%%%

\geometry{a4paper,vmargin=3cm,head=26.0pt}
\frenchspacing
\reversemarginpar
\hypersetup{
	colorlinks=true,
	linkcolor=blue,
	urlcolor=blue,
}
\renewcommand*{\thefootnote}{\fnsymbol{footnote}}
\pagestyle{scrheadings}
\KOMAoptions{
	headtopline = 1pt,
	headsepline = .6pt,
	footsepline = .6pt,
}
\setkomafont{pagehead}{\normalfont\small\sffamily}
\setkomafont{pagefoot}{\small\itshape}
\ihead{Einführung in das\\Textsatzsystem \LaTeX}
\chead{\bfseries Übungsblatt \blattnr}
\ohead{Ausgegeben: \ausgabetermin\\Abgabe: \abgabetermin}
\ifoot{Heidelberg, WS 2015}
\cfoot{}
\ofoot{Seite~\thepage~von~\pageref{LastPage}}

\MakeShortVerb{|}
\lstset{%
	backgroundcolor=\color[rgb]{.9 .9 .9},
	basicstyle=\ttfamily\small,
	breakindent=0em,
	breaklines=true,
	language=[LaTeX]TeX,
	commentstyle=,
	keywordstyle=,
	identifierstyle=,
	tabsize=2,
	captionpos=b,
	numbers=left,
	numberstyle=\tiny,
	stepnumber=0,
	numberfirstline=false,
	frame=single
}
\ifdraft
	\AtBeginDocument{\centerline{%
   	\color{red}\Large \sffamily Vorläufige Version}%
		\vspace*{1ex}%
	}
\fi
\newcommand{\abgabe}[1]{\par\noindent\textit{Abgabe:} #1}
\DeclareRobustCommand*\questionstar{\texorpdfstring{\bonusquestionsign}{* }}
\DeclareRobustCommand*\bonusquestionsign{\llap{$\bigstar$\space}}
\NewQuSolPair
	{question*}[name=\questionstar Bonusaufgabe]
	{solution*}[name=\questionstar Lösung]
\DeclareInstance{exsheets-heading}{myblock}{default}{
	title-pre-code = \sffamily,
	number-pre-code =  \sffamily\bfseries~\blattnr.,
	subtitle-pre-code = \upshape\sffamily\bfseries:~,
	points-pre-code = \itshape,
	join = {
		number[r,vc]subtitle[l,vc](0pt,0pt);
		title[r,vc]number[l,vc](0pt,0pt);
	} ,
	attach = { 
		main[l,vc]title[l,vc](0pt,0pt);
		main[r,vc]points[r,vc](0pt,0pt);
	},
}
\SetupExSheets{
	counter-format = qu[1],
	headings = myblock,
}
\SetupExSheets[points]{
	name = {\,Punkt/e},
	bonus-name = {Bonuspunkt/e},
}

\newcommand{\meta}[1]{\textcolor{gray}{$\langle$\texttt{\textsl{#1}}$\rangle$}}
\newcommand{\pkg}[1]{\href{http://ctan.org/pkg/#1}{\texttt{#1}}}
\newcommand{\TikZ}{Ti\textit{k}Z\xspace}

\usetikzlibrary{matrix}
\usetikzlibrary{fit}
\usetikzlibrary{backgrounds}
\usetikzlibrary{arrows}
\usetikzlibrary{shapes}
\DeclareMathOperator{\Coker}{Koker}
\DeclareMathOperator{\Ker}{Ker}


%%%%%%%%%%%%%%%%%%%%%%%%%%%%%%%%%%%%%%

\begin{document}


\begin{question}[subtitle=Daten darstellen mit pgfplots]{6}
	Bei einer Umfrage sind die in Tabelle \ref{tab:umfrage} dargestellten Daten erhoben worden. Diese Daten sollen Sie jetzt grafisch aufbereiten. 

	\begin{table}[h]
		\newcommand{\rth}[1]{\rotatebox{50}{\textsf{{#1}}}}
		\centering
		\begin{tabular}{l*{5}{S[table-number-alignment=left]}}
%			\toprule
			\textsf{{Frage}} & \rth{furchtbar} & \rth{meh} & \rth{ganz gut} & \rth{genial} & \rth{keine Ang.} \\
			\toprule
			Wie finden Sie Himbeereis? & 9 & 1 & 2 & 186 & 0 \\
			Mögen Sie Tanzen? & 32 & 63 & 52 & 49 & 2 \\
			Was halten Sie von Topf\/pflanzen? & 28 & 17 & 12 & 26 & 115 \\
			\bottomrule
		\end{tabular}
		\caption{Umfrage\/ergebnisse}
		\label{tab:umfrage}
	\end{table}
\noindent Nutzen Sie das Paket \pkg{pgfplots} um die Ergebnisse darzustellen. 
	Lassen Sie sich ruhig von der Paketdokumentation inspirieren und wählen Sie den Diagrammtyp oder die Diagrammtypen, die Sie für besonders geeignet halten. Je nach Darstellung können Sie dabei alle Daten in ein Diagramm eintragen, oder für jede Frage ein eigenes Diagramm erstellen.
	\abgabe{Quellcode per Mail{,} Quellcode und fertiges Dokument ausgedruckt.}
\end{question}

\newsavebox{\SolutionCodeA}
\begin{lrbox}{\SolutionCodeA}
\begin{minipage}{\textwidth}
\begin{LTXexample}[pos=b,rframe={}, preset=\centering]
\begin{tikzpicture}
  \begin{axis}[
    ybar,
    ymin=0,
    ymax = 200,
    x post scale = 1.5,
    enlarge x limits=0.1,
    symbolic x coords={furchtbar,meh,ganz gut,genial,keine Ang.},
    xtick={furchtbar,meh,ganz gut,genial,keine Ang.},
    legend style={
      at={(0.05,0.95)},
      anchor=north west
    },
  ]
    \addplot coordinates  {(furchtbar,9) (meh,1) (ganz gut,2) (genial,186) (keine Ang.,0)};
    \addlegendentry{Himbeereis}
    
    \addplot coordinates {(furchtbar,32) (meh,63) (ganz gut,52) (genial,49) (keine Ang.,2)};
    \addlegendentry{Tanzen}
    
    \addplot coordinates {(furchtbar,28) (meh,17) (ganz gut,12) (genial,26) (keine Ang.,115)};
    \addlegendentry{Topfpflanzen}
    
  \end{axis}
\end{tikzpicture}
\end{LTXexample}
\end{minipage}
\end{lrbox}

\begin{solution}
\noindent Für die Darstellung der Umfragedaten eignet sich zum Beispiel ein Säulendiagramm (|ybar|). Für die $x$-Achse würde ich die Antwort-Werte (furchtbar, meh, …) wählen. Da es sich dabei nicht um Zahlen handelt muss mit symbolischen Koordinaten gearbeitet werden. Legendeneinträge können mit \texttt{\textbackslash addlegendentry} hinzugefügt werden.

\noindent\usebox\SolutionCodeA
\end{solution}

\vspace{\baselineskip}
%\clearpage




\noindent Von den folgenden beiden Aufgaben müssen Sie nur eine bearbeiten! Die erste richtet sich vor allem an Mathematiker*innen, die zweite eher an Physiker*innen. Welche genau Sie bearbeiten steht Ihnen selbstverständlich frei. Sie können nur für eine Aufgabe Punkte erhalten, dafür gibt es aber bis zu sechs Bonuspunkte.\\

\begin{question}[subtitle=Schlangenlemma (Aufgabe für Mathematiker*innen)]{6+6}
Das Schlangenlemma\footnote{Wer nichts mit dem Begriff anfangen kann wird auf \href{https://de.wikipedia.org/wiki/Schlangenlemma}{Wikipedia} oder im \href{https://books.google.com/books?id=Fge-BwqhqIYC&pg=PA157}{Algebra-Buch} des Vertrauens fündig.} ist ein wichtiges Werkzeug in der homologischen Algebra, für das man kommutative Diagramme benutzt. Das in Abbildung \ref{fig:snake1} gezeigte Diagram wird im Schlangenlemma zum sogenannten Schlangendiagramm (Abbildung \ref{fig:snake2}) erweitert.

\begin{figure}[h]
	\centering
	\includegraphics{07snake1}
  \caption{kommutatives Diagramm mit exakten Zeilen}
  \label{fig:snake1}
\end{figure}

\begin{figure}[h]
	\centering
	\includegraphics{07snake2}
	\caption{Schlangendiagramm (Das Schlangenlemma hat seinen Namen vom Pfeil $D$, der sich wie eine Schlange durch das Diagramm windet)}
	\label{fig:snake2}
\end{figure}

\noindent Da jede Mathematiker*in wissen sollte, wie man kommutative Diagramme \TeX t soll das hier anhand des Schlangendiagramms geübt werden.
	\begin{enumerate}[label=\alph*)]
		\item Reproduzieren Sie das in Abbildung \ref{fig:snake2} gezeigte Diagramm in \LaTeX. Sie müssen sich dabei nicht an die hier verwendeten Pfeile und Farben, oder die Notation halten. Inhaltlich sollen sich die Diagramme aber entsprechen. Achten Sie darauf, dass mathematische Ausdrücke auch innerhalb der Abbildung im Mathemodus gesetzt werden.
		
		 Es gibt diverse Pakete, die Ihnen dabei die Arbeit erleichtern können. So lassen sich kommutative Diagramme zum Beispiel relativ elegant mit \TikZ\footnote{Hierfür eignet sich besonders die |matrix|-Library.} erzeugen.
		\item \emph{Bonusaufgabe:} Setzen Sie das komplette Schlangenlemma inklusive Beweis. %Sie können den Text dabei ruhig aus der e-Version eines Mathebuchs kopieren.
		
		Nutzen Sie dafür die \hologo{AmS}-Pakete und definieren Sie sich die nötigen Operatoren und Umgebungen mit |\textbackslash DeclareMathOperator| und |\textbackslash newtheorem| selbst. Die Darstellung der mathematischen Elemente im Text und in den Abbildungen (Diagrammen) sollen selbstverständlich gleich sein. 
	\end{enumerate}
	\abgabe{Quellcode per Mail{,} Quellcode und fertiges Dokument ausgedruckt.}
\end{question}

\newsavebox{\SolutionCodeB}
\begin{lrbox}{\SolutionCodeB}
\begin{minipage}{\textwidth}
% Danke an Fabian für die Lösung
\lstinputlisting{07snake2}
\end{minipage}
\end{lrbox}

\newpage
\begin{solution}
Das Diagramm in Abbildung 2 wurde mir folgendem Code erstellt:\\
\noindent\usebox\SolutionCodeB
\end{solution}



\begin{question}[subtitle=Zerfallsprozess (Aufgabe für Physiker*innen)]{6+6}
Sie haben den Zerfall eines Radioaktiven Isotops gemessen und müssen das Ergebnis nun grafisch darstellen. 
\begin{enumerate}[label=\alph*)]
\item Laden Sie sich die Datei \href{http://latexkurs.de/uebungen/07_messwerte.dat}{\texttt{07\_messwerte.dat}} von der Vorlesungshomepage herunter und stellen Sie die darin enthaltenen Daten mithilfe des Pakets \pkg{pgfplots} dar. Ordnen Sie dabei die Spalten wie folgt zu und stellen Sie sicher, dass auch der Messfehler im Diagramm zu sehen ist.
\begin{center}
\begin{tabular}{rl}
\toprule
\textbf{Spalte} & \textbf{Zuordnung}\\
\midrule
|zeit| & |x| \\
|zerfaelle| & |y| \\
|zerfaelle\_err| & |y error|\\
\bottomrule
\end{tabular}
\end{center}
\item \emph{Bonusaufgabe:} Nutzen Sie \LaTeX\ um die Zerfallskonstante $\lambda$ zu berechnen und zeichnen Sie die theoretische Kurve in das Diagramm zu den Messwerten. Welche Darstellung ist besonders geeignet um die mathematische Natur des Zerfallsgesetzes zu demonstrieren?

\end{enumerate}
	\abgabe{Quellcode per Mail{,} Quellcode und fertiges Dokument ausgedruckt.}
\end{question}
 

\newsavebox{\SolutionCodeC}
\begin{lrbox}{\SolutionCodeC}
\begin{minipage}{\textwidth}
\begin{LTXexample}[pos=b,rframe={},preset=\centering]
\begin{tikzpicture}
  \begin{axis}[
    xlabel={$t [\si{s}]$},
    ylabel={Anzahl Zerfälle},
    xmin=-10,
    xmax=610,
    legend cell align=left,
  ] 
    \addplot [
      blue,
      only marks,
      mark=.,
      error bars/.cd,
      y dir=both,
      y fixed relative=.1,
    ] table [
      x=zeit,
      y=zerfaelle,
    ] {07_messwerte.dat};
  \end{axis}
\end{tikzpicture}
\end{LTXexample}
\end{minipage}
\end{lrbox}

\newsavebox{\SolutionCodeD}
\begin{lrbox}{\SolutionCodeD}
\begin{minipage}{\textwidth}
\begin{lstlisting}
\addplot table [y={create col/linear regression={y=zerfaelle}}, mark=none]
  {07_messwerte.dat};
\end{lstlisting}
\end{minipage}
\end{lrbox}

\newpage

\begin{solution}
\noindent Daten aus externen Dateien lassen sich in \pkg{pgfplots} mit dem |table|-Befehl einbinden. Eine Mögliche Darstellung wäre die folgende:\\
\noindent\usebox\SolutionCodeC\\
\newpage
\noindent Für exponentielle Zusammenhänge eignet sich besonders eine logarithmische Darstellung, die man mit |semilogyaxis| anstatt |axis| erhält:
\begin{center}
\begin{tikzpicture}
  \begin{semilogyaxis}[
    xlabel={$t [\si{s}]$},
    ylabel={Anzahl Zerfälle},
    xmin=-10,
    xmax=610,
    legend cell align=left,
  ] 
    \addplot [
      blue,
      only marks,
      mark=.,
%      mark size=.5pt,
      error bars/.cd,
      y dir=both,
      y fixed relative=.1,
    ] table [
      x=zeit,
      y=zerfaelle,
    ] {07_messwerte.dat};
  \end{semilogyaxis}
\end{tikzpicture}
\end{center}
Mithilfe des Pakets \pkg{pgfplots-table} lassen sich aus den Tabellendaten leicht weitere Spalten errechnen. So kann man eine automatisch berechnete Ausgleichsgerade hinzufügen:\\

\noindent\usebox\SolutionCodeD
\end{solution}




\end{document}