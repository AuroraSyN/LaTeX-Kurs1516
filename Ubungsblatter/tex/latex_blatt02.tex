% !TEX TS-program = lualatex
% !TEX encoding = UTF-8 Unicode
% !TEX spellcheck = de_DE

\documentclass{scrartcl}

\usepackage{polyglossia}
\setmainlanguage{german}
\usepackage{fontspec}
\setsansfont{Linux Biolinum O}
\setromanfont{Linux Libertine O}
\setmonofont[Scale=.85]{Inconsolata}

\usepackage{
	enumitem,
	exsheets,
	exsheets-listings,
	geometry,
	hologo,
	lastpage,
	microtype,
	scrlayer-scrpage,
	shortvrb,
	hyperref,
}
\newif\ifdraft

%%%%%%%%%%%%%%%%%%%%%%%%%%%%%%%%%%%%%%
\newcommand{\blattnr}{2}
\newcommand{\ausgabetermin}{23.10.2015}
\newcommand{\abgabetermin}{30.10.2015}
\draftfalse
\SetupExSheets{solution/print=true}
%%%%%%%%%%%%%%%%%%%%%%%%%%%%%%%%%%%%%%

\geometry{a4paper,vmargin=3cm,head=26.0pt}
\frenchspacing
\reversemarginpar
\hypersetup{
	colorlinks=true,
	linkcolor=blue,
	urlcolor=blue,
}
\renewcommand*{\thefootnote}{\fnsymbol{footnote}}
\pagestyle{scrheadings}
\KOMAoptions{
	headtopline = 1pt,
	headsepline = .6pt,
	footsepline = .6pt,
}
\setkomafont{pagehead}{\normalfont\small\sffamily}
\setkomafont{pagefoot}{\small\itshape}
\ihead{Einführung in das\\Textsatzsystem \LaTeX}
\chead{\bfseries Übungsblatt \blattnr}
\ohead{Ausgegeben: \ausgabetermin\\Abgabe: \abgabetermin}
\ifoot{Heidelberg, WS 2015}
\cfoot{}
\ofoot{Seite~\thepage~von~\pageref{LastPage}}

\MakeShortVerb{|}
\lstset{%
	backgroundcolor=\color[rgb]{.9 .9 .9},
	basicstyle=\ttfamily\small,
	breakindent=0em,
	breaklines=true,
	language=[LaTeX]TeX,
	commentstyle=,
	keywordstyle=,
	identifierstyle=,
	tabsize=2,
	captionpos=b,
	numbers=left,
	numberstyle=\tiny,
	stepnumber=0,
	numberfirstline=false,
	frame=single
}
\ifdraft
	\AtBeginDocument{\centerline{%
   	\color{red}\Large \sffamily Vorläufige Version}%
		\vspace*{1ex}%
	}
\fi
\newcommand{\abgabe}[1]{\par\noindent\textit{Abgabe:} #1}
\DeclareRobustCommand*\questionstar{\texorpdfstring{\bonusquestionsign}{* }}
\DeclareRobustCommand*\bonusquestionsign{\llap{$\bigstar$\space}}
\NewQuSolPair
	{question*}[name=\questionstar Bonusaufgabe]
	{solution*}[name=\questionstar Lösung]
\DeclareInstance{exsheets-heading}{myblock}{default}{
	title-pre-code = \sffamily,
	number-pre-code =  \sffamily\bfseries~\blattnr.,
	subtitle-pre-code = \upshape\sffamily\bfseries:~,
	points-pre-code = \itshape,
	join = {
		number[r,vc]subtitle[l,vc](0pt,0pt);
		title[r,vc]number[l,vc](0pt,0pt);
	} ,
	attach = { 
		main[l,vc]title[l,vc](0pt,0pt);
		main[r,vc]points[r,vc](0pt,0pt);
	},
}
\SetupExSheets{
	counter-format = qu[1],
	headings = myblock,
}
\SetupExSheets[points]{
	name = {\,Punkt/e},
	bonus-name = {Bonuspunkt/e},
}

%%%%%%%%%%%%%%%%%%%%%%%%%%%%%%%%%%%%%%

\begin{document}
\begin{abstract}
	\noindent Geben Sie dieses Blatt (und alle folgenden Blätter) bitte in Gruppen von zwei bis drei Personen ab. Schreiben Sie Ihre Namen bitte als Kommentar (\texttt{\%}) in die Erste Zeile des Quellcodes jeder Aufgabe.
\end{abstract}

\begin{question}[subtitle=Schriften]{6}
	In der Vorlesung haben Sie zwei Methoden kennengelernt, um die Schriftart eines Dokumentes zu wechseln.
	Diese gilt es nun beide anzuwenden.
	\begin{enumerate}[label=\alph*)]
		\item Stöbern Sie ein wenig im „\href{http://www.tug.dk/FontCatalogue/}{LaTeX Font Catalogue}“ und suchen Sie sich eine Schrift aus, die Ihnen besonders gut gefällt. Demonstrieren Sie das Aussehen der Schrift in einem kurzen \hologo{pdfLaTeX}-Dokument. Nutzen Sie das Paket |blindtext|. Erzeugen Sie einen Absatz deutschsprachigen(!) Blindtext, in der von Ihnen gewählten Schriftart. Konsultieren Sie – falls nötig – die Paketdokumentationen (|texdoc|).
		\item In \hologo{XeLaTeX} und \hologo{LuaLaTeX} lassen sich – mittels des |fontspec|-Pakets – beliebige auf dem Betriebssystem installierte Schriften einsetzen. Liegt die Schrift im OpenType-Format (OTF) vor, können viele interessante Features, wie zum Beispiel kontextabhängige Buchstabenformen, genutzt werden.
		
		Die Schriftarten \emph{Linux Libertine} und \emph{Linux Biolinum} sind frei (im Sinne von Open Content) im Internet erhältlich. Laden Sie sich die OTF-Versionen der beiden Schriften von der \href{http://www.linuxlibertine.org/index.php?id=91&L=1}{Projektseite}\footnote{\url{http://www.linuxlibertine.org/}} und installieren Sie sie auf Ihrem Computer.
		
		Schreiben Sie ein kurzes Testdokument mit \hologo{XeLaTeX}, in dem beide Schriften vorkommen. Überlegen Sie sich dazu, welche Pakete Sie benötigen, welche Befehle nötig sind und welche Definitionen geschickt und sinnvoll sind. (Tipp: \texttt{\textbackslash setmainfont} u.\,ä.)
	\end{enumerate}
		\abgabe{Beide Quelltexte (nicht das PDF) per Mail und als Ausdruck.}
\end{question}
 

\newsavebox{\SolutionCodeA}
\begin{lrbox}{\SolutionCodeA}
\begin{minipage}{.93\textwidth}
\begin{lstlisting}
\documentclass{minimal}

\usepackage[ngerman]{babel}
\usepackage{tgpagella}
\usepackage[T1]{fontenc}

\begin{document}

\blindtext

\end{document}
\end{lstlisting}
\end{minipage}
\end{lrbox}

\newsavebox{\SolutionCodeB}
\begin{lrbox}{\SolutionCodeB}
\begin{minipage}{.93\textwidth}
\begin{lstlisting}
\documentclass{scrartcl}
\usepackage{xltxtra}
\usepackage[german]{polyglossia}

\setmainfont{Linux Libertine O}
\setsansfont{Linux Biolinum O}
\begin{document}
\section{Serifenlose eignen sich gut für Überschriften}
Serifen sollen die das lesen erleichern, indem Sie den Augen als Linie dienen. Deshalb eignen sich Serifenschriften besonder gut als Brotschrift.
\end{document}
\end{lstlisting}
\end{minipage}
\end{lrbox}


\begin{solution}
	\begin{enumerate}[label=\alph*)]
		\item Unter \hologo{pdfLaTeX} kann man Schriften durch Laden der entsprechenden Pakete nutzen:
		\\\usebox\SolutionCodeA

		\item {Mit \hologo{XeLaTeX} oder \hologo{LuaLaTeX} sollte fontspec verwendet werden: 
		\\\usebox\SolutionCodeB
		}
	\end{enumerate}
\end{solution}


\begin{question}[subtitle=Liste mit Umlauten]{6}
	Im Folgenden sollen Sie ein Dokument mit einer beliebigen \TeX-Maschine\footnote{Sie können die Datei wahlweise mit |pdflatex|, |xelatex| oder |lualatex| setzen. Geben Sie das gewählte Programm bitte als Kommentar (\texttt{\%}) in der zweiten Zeile des Quellcodes an.} setzen.
	\begin{enumerate}[label=\alph*)]
		\item Erstellen Sie ein Dokument der Klasse |scrartcl|, das in der Kopfzeile links die Nummer dieser Aufgabe und rechts Ihre Namen enthält.
		\item Der Quellcode soll Umlaute (|äöüß|) enthalten, die im PDF auch korrekt als Umlaute ausgegeben werden. Überlegen Sie sich, welche Pakte Sie – in Abhängigkeit der gewählten \TeX-Maschine – laden müssen und laden Sie nur die Pakete die Sie wirklich benötigen.
		\item Fertigen Sie im Dokument eine nummerierte Liste an, in der Sie (nach Beliebtheit sortiert) Ihre Lieblingstiere aufzählen. Achten Sie darauf, dass im Dokument auch tatsächlich Umlaute vorkommen (nutzen Sie zum Beispiel Verniedlichungen).
	\end{enumerate}
		\abgabe{Quelltext per Mail und als Ausdruck.}
\end{question}


\newsavebox{\SolutionCodeC}
\begin{lrbox}{\SolutionCodeC}
\begin{minipage}{\textwidth}
Mit \hologo{XeLaTeX} ließe sich die Aufgabe beispielsweise so lösen:
\begin{lstlisting}
\documentclass{scrartcl}

\usepackage{polyglossia} 
\setmainlanguage{german}
\usepackage{xltxtra}

\lohead{Aufgabe 2.2}
\rohead{Meier, Müller}
\pagestyle{scrheadings}

\begin{document}

\begin{enumerate}
  \item Hündchen
  \item Kätzchen
  \item Mäuschen
\end{enumerate}

\end{document}
\end{lstlisting}
\end{minipage}
\end{lrbox}


\begin{solution}
	\usebox\SolutionCodeC
\end{solution}

\end{document}