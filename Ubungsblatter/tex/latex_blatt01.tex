% !TEX TS-program = lualatex
% !TEX encoding = UTF-8 Unicode
% !TEX spellcheck = de_DE

\documentclass{scrartcl}

\usepackage[ngerman]{babel}
\usepackage{fontspec}
\setsansfont{Linux Biolinum O}
\setromanfont{Linux Libertine O}
\setmonofont[Scale=.85]{DejaVu Sans Mono}

\usepackage{
	enumitem,
	exsheets,
	exsheets-listings,
	geometry,
	lastpage,
	microtype,
	scrlayer-scrpage,
	shortvrb,
	hyperref,
}
\newif\ifdraft

%%%%%%%%%%%%%%%%%%%%%%%%%%%%%%%%%%%%%%
\newcommand{\blattnr}{1}
\newcommand{\ausgabetermin}{16.10.2015}
\newcommand{\abgabetermin}{23.10.2015}
\draftfalse
\SetupExSheets{solution/print=false}
%%%%%%%%%%%%%%%%%%%%%%%%%%%%%%%%%%%%%%

\geometry{a4paper,vmargin=3cm,head=26.0pt}
\frenchspacing
\reversemarginpar
\hypersetup{
	colorlinks=true,
	linkcolor=blue,
	urlcolor=blue,
}
\renewcommand*{\thefootnote}{\fnsymbol{footnote}}
\pagestyle{scrheadings}
\KOMAoptions{
	headtopline = 1pt,
	headsepline = .6pt,
	footsepline = .6pt,
}
\setkomafont{pagehead}{\normalfont\small\sffamily}
\setkomafont{pagefoot}{\small\itshape}
\ihead{Einführung in das\\Textsatzsystem \LaTeX}
\chead{\bfseries Übungsblatt \blattnr}
\ohead{Ausgegeben: \ausgabetermin\\Abgabe: \abgabetermin}
\ifoot{Heidelberg, WS 2015}
\cfoot{}
\ofoot{Seite~\thepage~von~\pageref{LastPage}}

\MakeShortVerb{|}
\lstset{%
	backgroundcolor=\color[rgb]{.9 .9 .9},
	basicstyle=\ttfamily\small,
	breakindent=0em,
	breaklines=true,
	language=[LaTeX]TeX,
	commentstyle=,
	keywordstyle=,
	identifierstyle=,
	tabsize=2,
	captionpos=b,
	numbers=left,
	numberstyle=\tiny,
	stepnumber=0,
	numberfirstline=false,
	frame=single
}
\ifdraft
	\AtBeginDocument{\centerline{%
   	\color{red}\Large \sffamily Vorläufige Version}%
		\vspace*{1ex}%
	}
\fi
\newcommand{\abgabe}[1]{\par\noindent\textit{Abgabe:} #1}
\DeclareRobustCommand*\questionstar{\texorpdfstring{\bonusquestionsign}{* }}
\DeclareRobustCommand*\bonusquestionsign{\llap{$\bigstar$\space}}
\NewQuSolPair
	{question*}[name=\questionstar Bonusaufgabe]
	{solution*}[name=\questionstar Lösung]
\DeclareInstance{exsheets-heading}{myblock}{default}{
	title-pre-code = \sffamily,
	number-pre-code =  \sffamily\bfseries~\blattnr.,
	subtitle-pre-code = \upshape\sffamily\bfseries:~,
	points-pre-code = \itshape,
	join = {
		number[r,vc]subtitle[l,vc](0pt,0pt);
		title[r,vc]number[l,vc](0pt,0pt);
	} ,
	attach = { 
		main[l,vc]title[l,vc](0pt,0pt);
		main[r,vc]points[r,vc](0pt,0pt);
	},
}
\SetupExSheets{
	counter-format = qu[1],
	headings = myblock,
}
\SetupExSheets[points]{
	name = {\,Punkt/e},
	bonus-name = {Bonuspunkt/e},
}

%%%%%%%%%%%%%%%%%%%%%%%%%%%%%%%%%%%%%%

\begin{document}
\begin{abstract}
	\noindent Achtung: Da die Installation der \TeX-Distribution grundlegend für den Kurs ist,
	muss die Abgabe für dieses Blatt von jedem einzeln bearbeitet werden. 
	\emph{Keine Gruppenabgabe!}
\end{abstract}

\begin{question}[subtitle=Minimales \TeX-Dokument]{12}
	Grundlage des \LaTeX-Kurses ist eine funktionsfähige \TeX-Distribution. 
	\begin{enumerate}[label=\alph*)]
		\item Installieren Sie ein lauffähiges \TeX-System\footnote{ Es steht Ihnen frei, \TeX~Live, MiK\TeX, Mac\TeX~oder Pro\TeX t zu installieren, solange die \TeX-Distribution nicht älter als ein Jahr ist. In der Vorlesung wird von einer \href{http://www.tug.org/texlive/}{\TeX~Live 2015}-Installation ausgegangen.} auf Ihrem Rechner. Konsultieren Sie hierzu die \href{http://latexkurs.de/uebungen/texlive_anleitung.pdf}{Anleitung auf der Vorlesungshomepage}. Machen Sie sich mit dem System vertraut, testen Sie verschiedene Befehle …

		\item Erstellen Sie nun ein minimales plain\TeX-Dokument. Verwenden Sie dazu einen Texteditor und kompilieren Sie über die Kommandozeile! \\(Befehl: |pdftex mydocument.tex|)

		Außer normalem Text soll nur ein einziges \TeX-Kommando verwendet werden (welches und warum genau dieses?). Schreiben Sie |ä,ö,ü,ß| als |ae,oe,ue,ss| und verfassen Sie einen kurzen Text (zwei bis drei Sätze) darüber, welche Themen Sie gerne in der Vorlesung behandeln würden.

		\item Erstellen Sie weiterhin ein minimales \LaTeX-Dokument, das mindestens „Hallo Welt!“ in eine pdf-Datei ausgibt. Die Wahl der \TeX-Maschine ist dabei Ihnen überlassen.
	\end{enumerate}
	\abgabe{Beide Quelltexte per Mail und das fertige \TeX-Dokument (das PDF von Teil b) als Ausdruck.}
\end{question}


\newsavebox{\SolutionCodeB}
\begin{lrbox}{\SolutionCodeB}
\begin{minipage}{.93\textwidth}
\begin{lstlisting}
Ich wuerde gerne viel ueber den Mathesatz, ueber
Tabellen und das Einfuegen von Bildern lernen.
Ausserdem bin ich sehr an typografischen Feinheiten
wie Microtypographie interessiert.

\bye
\end{lstlisting}
\end{minipage}
\end{lrbox}

\newsavebox{\SolutionCodeC}
\begin{lrbox}{\SolutionCodeC}
\begin{minipage}{.93\textwidth}
\begin{lstlisting}
\documentclass{minimal}
\begin{document}
	Hallo Welt!
\end{document}
\end{lstlisting}
\end{minipage}
\end{lrbox}


\begin{solution}
	\begin{enumerate}[label=\alph*)]
		\item Es muss der Befehl \texttt{\textbackslash bye} oder \texttt{\textbackslash end} vorkommen, damit das Dokument beendent wird. Ansonsten ist in plainTeX kein weiterer Befehl nötig. Als Ausgabe erhält man die Datei \verb|mydocument.pdf|, wenn die Datei selbst \verb|mydocument.tex| hieß. Hätte man mit dem Befehl |tex| kompiliert, hätte man eine |dvi|-Datei erhalten, die oft vor dem Drucken erst umgewandelt werden muss. Beispiel:
		\\\noindent\usebox\SolutionCodeB

		\item Hierfür muss etwas mehr Aufwand betrieben werden: Die Dokumentenklasse muss definiert sein und eine |document|-Umgebung muss verwendet werden:
		\\\noindent\usebox\SolutionCodeC\\
	\end{enumerate}
\end{solution}

\iffalse
\begin{question*}[subtitle=Übungsblatt selbst erstellen]{+0}
	Laden Sie sich den Sourcecode dieses Übungsblattes von der Vorlesungshomepage herunter und kompilieren sie das Dokument, mit |pdflatex|, selbst.
	\abgabe{keine}
\end{question*}
\begin{solution*}
	Das generierte PDF mit diesem Text ist die Lösung der Aufgabe.
\end{solution*}
\fi

\end{document}