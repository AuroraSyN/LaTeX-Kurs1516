% Dorkenwald, Soloninov
% Blatt 4, Aufgabe 1;

\documentclass [12pt]{scrartcl}
\usepackage[ngerman]{babel} 			% Deutsche Sprachanpassungen
\usepackage[T1]{fontenc}    			% Silbentrennung bei Sonderzeichen
\usepackage[utf8]{inputenc} 			% Direkte Angabe von Umlauten im Dokument.
\usepackage{booktabs}			
\usepackage{multirow}

\begin{document}
 \begin{table}[htbp]
  \begin{center}
    \caption{Gemüse vegfügbarkeit}
    \centering

    % 5 links, 1 rechts
    \begin{tabular}{ *{5}{l} *{1}{r}}
	\toprule

	\multicolumn{1}{l}{Produkt} & \multicolumn{1}{l}{Herkunft}  
	& \multicolumn{1}{l}{Saisonbegin} & \multicolumn{1}{l}{Saisonende} 
	& \multicolumn{1}{l}{Handelsklasse} & \multicolumn{1}{r}{vegfürbar} \\
	
	\cmidrule(lr){1-6}
	Auberginen & Frankreich & Juli & September & I & -  \\
	\cmidrule(lr){1-6}
	Esskastanien & Frankreich & September & September & I & - \\ 
	\cmidrule(lr){1-6}
	Feldsalat & Deutschland & Oktober & Februar & II & ja \\
	\cmidrule(lr){1-6}
	Kürbis & Deutschland & August & Dezember & I & ja \\
	\cmidrule(lr){1-6}
	Rote Beete & Italien & September & Februar & I  & ja \\
	\cmidrule(lr){1-6}
	Zucchini & Spanien & Juni & Oktober & II & - \\
	\cmidrule(lr){1-6}
	Zwiebeln & Deutschland & Mai & Oktober & -  & - \\
	\bottomrule
    \end{tabular}

  \end{center}
 \end{table}

\end{document}